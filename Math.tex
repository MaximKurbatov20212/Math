\documentclass[12pt, english]{article}
\usepackage{scrextend}
\usepackage{graphicx}
\usepackage{amssymb}
\usepackage{xcolor}
\usepackage{hyperref}
\usepackage{amsmath}
\usepackage{wrapfig}
\usepackage{tocloft}
\definecolor{linkcolor}{HTML}{002f55} % цвет ссылок
\definecolor{urlcolor}{HTML}{002f55} % цвет гиперссылок
\hypersetup{pdfstartview=FitH,  linkcolor=linkcolor,urlcolor=urlcolor, colorlinks=true}
\changefontsizes[20pt]{14pt}
\usepackage[utf8]{inputenc}
\usepackage[left=20mm, top=20mm, right=10mm, bottom=20mm,  nohead,nofoot]{geometry}

\usepackage[russian]{babel}

\begin{document}


\tableofcontents
\setcounter{secnumdepth}{1}
\setcounter{tocdepth}{1}
\newpage

\part{Интегрирование на многообразиях}
\section{Кривые в $R^n$}
Мы будем рассматривать наши кривые в пространстве $R^n$. Иногда в формулировке теоремы или утверждения нет условия на непрерывность кривой. Это не означает, что его нет, возможно оно и так подразумевается и без него утверждение становится интуитивно некорректным.
\subsection*{Определение 1.1:}
\begin{flushleft}
\textbf{Непрерывная кривая} — множество точек
$\varphi_1(t), ... , \varphi_n(t), t \in [a,b]$

$A = \varphi_1(a), ... , \varphi_n(a)$

$B = \varphi_1(b), ... , \varphi_n(b)$

Если A = B, то кривая замкнута.
\end{flushleft}
\subsection*{Определение 1.2:}
\begin{flushleft}
$\Phi(t) = (\varphi_1(t), ... , \varphi_n(t))$ — параметризация кривой

\textbf{Важный факт:} существует бесконечное кол-во способов параметризовать кривую
\end{flushleft}
\subsection*{Определение 1.3:}
\begin{flushleft}
Если для кривой выполнятеся:
$\exists \varphi_1'(t), ... , \varphi_n'(t)$ такие, что

$\varphi_1'^2(t)+ ... +\varphi_n'^2(t) > 0, t \in [a,b]$, то такую кривую называем \textbf{гладкой}

Если $\varphi_1'^2(t)+ ... +\varphi_n'^2(t) = 0,$ при $t = m$, то такая точка \textbf{особенная}
\end{flushleft}
\subsection*{Определение 1.4:}
\begin{flushleft}
\textbf{Кусочно-гладкая кривая} — \textbf{непрерывная} гладкая 
кривая, состоящая из \textbf{конечного} числа гладких кривых.

\textbf{Важный факт:} не каждая кривая является спрямляемой
\end{flushleft}
\subsection*{Определение 1.5:}
\begin{flushleft}
\textbf{Спрямляемая кривая} — кривая, имеющая конечную длину.

\textbf{Важный факт:} гладкая кривая всегда спрямляемая

\end{flushleft}
\subsection*{Определение 1.6:}
\begin{flushleft}
\textbf{Натуральная параметризация} — параметризация, параметром которой выступает длина \textbf{дуги} от начала до точки на кривой.

Обозначаем ее как $\Psi(s)$ , где $s$ — длина дуги
\end{flushleft}
\begin{figure}[h]
\centering
\includegraphics[width = 8cm]{кривая.jpg}
\end{figure}

\subsection*{Теорема 1.1:}
	Для любой гладкой кривой существует натуральная параметризация.
	
	\textit{Без доказательства.}
	
\subsection*{Любопытное утверждение:}
Если кривая гладкая и без особых точек с гладкой параметризацией 
$\Phi(t)$ и натуральной параметризацией $\Psi(s)$ справедливо:
\begin{center}
$\frac{ds}{dt} = |\Phi'(t)|$
\end{center}
\newpage
\subsection*{Некоторые факты:}
Задание параметризации $(\varphi_1(t), ... , \varphi_n(t))$ определяет движение на кривой $\Gamma $от ее начальной точки к конечной, или, другими словами, определяет ориентацию кривой, называемую положительной. Если при переходе от исходной параметризации начальная и конечная точки меняются местами (в случае замкнутой кривой — меняется направление движения), то происходит смена ориентации от положительной к отрицательной. Кривую $\Gamma$ с положительной по отношению к исходной параметризации ориентацией обозначают $\Gamma^+$, с отрицательной — $\Gamma^-$.


\section{Криволинейные интегралы I рода}
\subsection*{Определение 1.7:}
	Пусть задана гладкая, спрямляемая кривая с параметризацией 
	$\Phi(t)$
	
	$\Gamma: \Phi(t) , t \in [a,b]$
	
	Также есть натуральная параметризация:
	
	 $\Gamma: \Psi(s),$
	$s \in [0 , S_\Gamma]$, в силу спрямляемости
	
	И пусть задана функция $F(x) , x \in \Gamma$
	
	Тогда \textbf{криволинейным интегралом I рода от $F$ по $\Gamma$}
	назовем интеграл Римана:
	
	$$ I = \int\limits_{0}^{S_\Gamma} F(\Psi(s)) \,ds  = 
	 \int\limits_{0}^{S_\Gamma} 	F(s) \,ds $$
	И будем обозначать его, как 
 	$$I = \int\limits_{\Gamma}^{} F_0(x) \,ds$$
\newpage
\subsection{Свойства криволинейных интегралов I рода}
\subparagraph{Свойство 1:} $F(s) = 1 \Rightarrow I = S_\Gamma$

\textit{Док-во:}
	$$F(s) = 1 \Rightarrow I = \int\limits_{0}^{S_\Gamma} 1 \,ds \Rightarrow
	I = S_\Gamma - 0 = S_\Gamma$$ читд

\subparagraph{Свойство 2:} Криволинейный интеграл I рода не зависит от ориентации кривой, те $$\int\limits_{\Gamma^+}^{} F_0(x) \,ds = \int\limits_{\Gamma^-}^{} F_0(x) \,ds$$

\textit{Док-во:}


	Пусть дана кривая с натуральной параметризацией $\Psi(s),s \in [0 , S_\Gamma]$:
	
	$\Gamma^+: A = \Psi(0) , B = \Psi(S_\Gamma)$
	
	Возьмем точку $M \in [A , B]$ на кривой, тогда $M = \Psi(s)$
	
	Определим параметр $\sigma = S_\Gamma - s$, те $\sigma$ — расстояние от $B$ до $M$.Тогда

	$$\int\limits_{\Gamma^+}^{} F_0(x) \,ds = 
	\int\limits_{0}^{S_\Gamma} F(\Psi(s)) \,ds 
	\stackrel{\sigma = S_\Gamma - s}{=} 
	- \int\limits_{S_\Gamma}^{0} F(\Psi(\sigma - S_\Gamma)) \,d\sigma =  $$		 	
	$$=\int\limits_{0}^{S_\Gamma}F(\Psi(\sigma - S_\Gamma)) \,d\sigma = 
	\int\limits_{\Gamma^-}^{} F_0(x) \,d\sigma$$
Тк криволинейный интеграл I рода не зависит от выбранной параметризации, то свойство 2 доказано.
читд

\subparagraph{Свойство 3:}
Пусть $\Gamma$ есть кривая в $R^n$ с непрерывно дифференцируемой
на отрезке $[a, b$] параметризацией $\Phi(t)$ без особых точек, тогда справедливо равенство
$$\int\limits_{\Gamma}^{} F_0(x) \,ds = \int\limits_{a}^{b} F(\Phi(t))[\varphi_1'^2(t)+ ... +\varphi_n'^2(t)] ^ \frac{1}{2} \,dt $$
\textit{Без доказательства}

\subparagraph{Свойство 4:}
Пусть $\tau$ = $\{s_i\}_{i = 0}^{m}$ есть разбиение отрезка $[0, S_\Gamma]$, $\xi_i$  есть
точки из отрезков $[s_{i-1}, s_i], i = 1,...,m, \Delta s_i = s_i - s_{i-1}$  длина дуги
кривой $\Gamma$ от точки $\Psi_0(s_{i-1})$ до точки $\Psi_0(s_i)$, $\sigma_\tau$ — интегральная сумма функции $F_0(s)$ по отрезку $[0, S_\Gamma]$

$$ \sigma_\tau = \sum_{i=1}^m F_0(\Psi_0(\xi_i))\Delta s_i$$

Тогда, если криволинейный интеграл I первого рода существует, то
	\[ \lim_{max(\Delta s_i)\to 0} \sigma_\tau  = I \]
\textit{Док-во:}

	Вспомним, как мы определяли интеграл Римана. Мы составляли интегральные суммы, потом устремляли разбиение к нулю и говорили, если вот существует такой предел, то назовем его интегралом Римана. Тут у нас условие, что криволинейный интеграл I первого рода существует, значит существует интеграл Римана, значит и предел сумм есть, который как раз и равен нашему интегралу Римана.

\subparagraph{Свойство 5:}
Если функция $F(x)$ представляет собой комбинацию
$\alpha F_1(x) + \beta F_2(x)$, $\alpha, \beta$ — фиксированные числа, криволинейные интегралы по кривой $\Gamma$ от функций $F_1(x)$ и $F_2(x$) существуют, то выполняется
равенство.

	$$\int\limits_{\Gamma}^{} F_0(x) \,ds 
	= 
	\alpha \int\limits_{\Gamma}^{} F_1(x) \,ds + 
	\beta \int\limits_{\Gamma}^{} F_2(x) \,ds $$

\textit{Док-во:}
$$
	\int\limits_{\Gamma}^{} F_0(x) \,ds
=
 	\int\limits_{0}^{S_\Gamma} F(\Psi(s)) \,ds 
=
	\int\limits_{0}^{S_\Gamma} \alpha F_1(\Psi(s))+ \beta F_2(\Psi(s))\,ds =
$$
$$
	\int\limits_{0}^{S_\Gamma} \alpha F_1(\Psi(s)) \,ds + 
	\int\limits_{0}^{S_\Gamma} \beta F_2(\Psi(s)) \,ds 
= 
	\alpha \int\limits_{0}^{S_\Gamma}  F_1(\Psi(s)) \,ds +  
	\beta \int\limits_{0}^{S_\Gamma}  F_2(\Psi(s)) \,ds 
=
$$
$$
	\alpha \int\limits_{\Gamma}^{}  F_1(x) \,ds +  
	\beta \int\limits_{\Gamma}^{}  F_2(x) \,ds 
$$
читд
	
	Вообщем сводим криволинейный интеграл к интегралу Римана, а там эти свойства уже доказаны в прошлом семестре.

	
\subsection*{Определение 1.8:}	
\textbf{Криволинейным интегралом по кусочно-гладкой кривой} $\Gamma$ называется число

\begin{equation}\label{eq1}
\int\limits_{\Gamma_1}^{} F_0(x)\,ds  + \int\limits_{\Gamma_2}^{} F_0(x)\,ds
\end{equation}

если каждый из криволинейных интегралов по $\Gamma_1$ и $\Gamma_2$ существуют.
	
	
\textbf{Замечание} Поскольку понятие определенного интеграла
 $$\int\limits_{a}^{b} F(x)\,dx $$
по отрезку можно расширить — например, до несобственного интеграла от
неограниченных функций или по неограниченному промежутку — то и понятие криволинейного интеграла первого рода можно расширить, определив несобственный криволинейный интеграл первого рода, или же перейти
к какой-либо иной конструкции, расширяющей понятие обычного определенного интеграла.	
	\newpage	
	
\section{Криволинейные интегралы II рода}	
Пусть $\Gamma$ есть кривая, параметризованная непрерывно-дифференцируемой
на отрезке $[a, b]$ вектор-функцией $\Phi(t)$, и пусть эта кривая не имеет особых
точек. Тогда, во-первых, в каждой точке $\Phi(t)$ определена касательная к  $\Gamma$,
и, во-вторых, от параметризации $\Phi(t)$ можно перейти к эквивалентной ей
натуральной параметризации $\Psi(s$). Обозначим через $cos \alpha_k, k = 1,...,n,$
направляющие косинусы единичного вектора $\vec{l} = \vec{l}(t)$ касательной к $\Gamma$ в
текущей точке (другими словами, искомый вектор $\vec{l}$ задается равенством 
$\vec{l} = (cos \alpha_1,..., cos \alpha_n) $и $\alpha_k, k = 1,...,n,$ есть углы между вектором $\vec{l}$
и положительным направлением соответствующей оси $Ox_k)$. 

\subsection*{Определение 1.9:}	
	Пусть задана функция $F_0(x)$, определенная при $x \in \Gamma$, 

и пусть $ F(s)$ = $F_0(\Psi(s))$

Тогда \textbf{криволинейным интегралом второго рода по кривой $\Gamma$ от функции $F_0(x)$} по координате $x_k,$ $ k = 1,...,n,$ называется интеграл:

	$$I = \int\limits_{\Gamma}{} F  cos \alpha_k d{s},$$
	если последний существует
	
	Обозначают как:
	
		$$I = \int\limits_{\Gamma}{} F d{x_k},$$
	\newpage
\subsection*{Определение 1.10:}
	Область $G$ из пространства $R^2$ называется \textbf{элементарной относительно оси $Oy$}, если ее граница состоит из графиков двух непрерывных функций $\phi(x)$ и $\psi(x)$, определенных при
$x \in [a, b]$ и таких, что $\phi(x)	\leq \psi(x)$ для всех $x$, а также, быть
может, из некоторых отрезков прямых $x = a$ и $x = b$.


\includegraphics[width = 8cm]{oy.jpg}
\includegraphics[width = 8cm]{oy_1.jpg}

\subsection*{Определение 1.11:}
	Область $G$ из пространства $R^2$ называется \textbf{элементарной относительно оси $Ox$} областью, если ее граница состоит из графиков двух непрерывных функций $\alpha(y)$ и $\beta(y)$, определенных при
$y \in [c, d]$ и таких, что $\alpha(y)	\leq \beta(y)$ для всех $y$, а также, быть
может, из некоторых отрезков прямых $y = c$ и $y = d$.


\includegraphics[width = 8cm]{ox.jpg}
\includegraphics[width = 8cm]{ox_1.jpg}
\newpage
\subsection*{Замечание:}
	Группы Шваб идут по Крючковичу, у которого такие области называются \textbf{простыми} и, к тому же, оси меняются местами.
	
	
\includegraphics[width = 18cm]{крючкович_x.jpg}

\includegraphics[width = 18cm]{крючкович_y.jpg}
\newpage
\subsection{Формула Грина:}
	Пусть $D$ есть ограниченная область из пространства $R^2$
с кусочно-гладкой границей $\Gamma$, ориентированной положительно, и пусть
эту область можно разбить на конечное число непересекающихся элементарных областей с кусочно-гладкими положительно-ориентированными границами. Далее, пусть $P(x, y)$ и $Q(x, y)$ есть заданные функции
такие, что 

1)  $P$ и $Q$ непрерывны в замкнутой области $D$


2) $P$ и $Q$ имеют непрерывные частные производные $\frac{\partial Q}{\partial x},\frac{\partial P}{\partial y} $ в замкнутой $D$

тогда верна формула Грина:

\begin{equation}\label{eq3}
	\int\limits_D \int (\frac{\partial Q}{\partial x} - \frac{\partial P}{\partial y}) dxdy =
	\oint\limits_{\Gamma} (Pdx + Qdy)
\end{equation}
\textit{Док-во для элементарной(и по $Ox$, и по $Oy$) $D$:} 

	Сведем двойной интеграл к повторному и применим формулу Ньютона-Лейбница:
$$
\int\limits_D \int \frac{\partial Q}{\partial x} dxdy = 
\int\limits_{c}^d dy \int\limits_{\alpha(y)}^{\beta(y)}\frac{\partial Q}{\partial x}dx = 
\int\limits_{c}^d  Q(x,\beta(y) - Q(x,\alpha(y))dy =
$$

$$
\int\limits_{c}^d  Q(x,\beta(y))dy -\int\limits_{c}^d Q(x,\alpha(y))dy
$$  
\begin{wrapfigure}{r}{9cm}
\includegraphics[width = 9cm]{грин.jpg}	
\end{wrapfigure}

Мы можем параметризовать наши кривые 

	$
	\gamma_1: \alpha(t) , t \in [c , d].
	$
	
	$
	\gamma_2: \beta(t) , t \in [c , d].
	$
	
	$
	\gamma_3:  y = c, x = t , t \in [\alpha(c) , \beta(c)].
	$
	
	$
	\gamma_4:  y = d, x = t , t \in [\alpha(d) , \beta(d)].	
	$
\newpage




Перепишем наши интегралы как криволинейные. Не забываем, что есть разница в направлении кривой!!!

$$
\int\limits_{\gamma_1}  Q(x,y)dy -\int\limits_{\gamma_2^-} Q(x,y)dy
= \int\limits_{\gamma_1}  Q(x,y)dy + \int\limits_{\gamma_2} Q(x,y)dy
$$  
Заметим, что
$$ 
\int\limits_{\gamma_3}  Q(x,y)dy = \int\limits_{\gamma_4} Q(x,y)dy = 0
$$
У нас получился интеграл по замкнутому контуру
$$ 
\int\limits_{\gamma_1}  Q(x,y)dy + \int\limits_{\gamma_2} Q(x,y)dy +
\int\limits_{\gamma_3}  Q(x,y)dy = \int\limits_{\gamma_4} Q(x,y)dy = 
\oint\limits_{\Gamma} Qdy
$$
$$
\int\limits_D \int\frac{\partial Q}{\partial x}dxdy = 
\oint\limits_{\Gamma} Qdy
$$

Аналогично:
$$
\int\limits_D \int\frac{\partial P}{\partial y}dxdy = 
-\oint\limits_{\Gamma} Pdx
$$
Складывая, получаем:
$$
\int\limits_D \int( \frac{\partial Q}{\partial x} - \frac{\partial P}{\partial y} )dxdy =
\oint\limits_{\Gamma} (Pdx + Qdy)
$$
читд


\textit{Док-во, если $D$ состоит из мн-ва непересекающихся, ненулевых элементарных областей:}

$
D = D_1 \cup D_2 \cup ... \cup D_m $ при $ D_i \cap D_j \neq  \varnothing, i \neq j
$
В силу свойства аддитивности двойного интеграла и факта, что граница области имеет нулевую меру:
$$
\int\limits_D \int (\frac{\partial Q}{\partial x} - \frac{\partial P}{\partial y}) dxdy
=
\sum_{i=1}^{m} \int\limits_{D_i} \int (\frac{\partial Q}{\partial x} - \frac{\partial P}{\partial y}) dxdy
$$
Применяя теперь для каждого слагаемого в правой части данного равенства доказанную выше формулу, получим
$$
\int\limits_D \int (\frac{\partial Q}{\partial x} - \frac{\partial P}{\partial y}) dxdy
= 
\sum_{i=1}^{m} \oint\limits_{D_i} (Pdx + Qdy) 
$$
В сумме, стоящей справа, содержатся интегралы по положительно ориентированным частям границ областей $D_i$, составляющим в целом границу $D$,
а также содержатся интегралы по тем частям границ областей  $D_i$, которые
лежат внутри  $D$, причем эти интегралы берутся дважды по одинаковым
кривым, но с противоположной ориентацией — в силу свойств криволинейных интегралов второго рода они взаимно уничтожаются. В результате
суммирования как раз и получится требуемое равенство.

читд
\subsection*{Замечание:}
Может возникнуть вопрос, что это за странная запись такая? 
$$
\oint\limits_{\Gamma}(Pdx + Qdy)
$$

Ведь у нас никогда не было, что разные функции интегрируются по разным переменным в одном интеграле. Можно это понимать так: Мы хотим вычислить силу, поэтому интегрируем работу по составляющим, где $P$ $x$-составляющая, $Q$ $y$-составляющая.
	
Или просто воспринимайте его как сумму:
$$
\oint\limits_{\Gamma} (Pdx + Qdy) =
\oint\limits_{\Gamma}Pdx + \oint\limits_{\Gamma} Qdy
$$

\subsection*{Замечание:}
	Не обязательно писать именно интеграл по замкнутой кривой, можно просто интеграл. Просто два нулевых интеграла нам дают такую возможность.
\newpage

\subsection*{Определение 1.12:}
	Зададим $(m+1)$ гладкие, замкнутые кривые $\Gamma_0 , \Gamma_1$ ...$\Gamma_m $
	
	Пусть $\Gamma_0$ - граница области $G$ 
	и $\Gamma_i \cap \Gamma_j = \varnothing$ при $ i \neq j$
	
	$\Gamma_i$ - граница области $G_i$, $\Gamma_i \in G$ ,  $i = 1$ ...$ m$
	
	Тогда $G \setminus (G_1 \cup G_2 \cup$...$\cup G_m)$ - $(m+1)$-связная область
	
\includegraphics[width = 10cm]{связная_область.jpg}

Заметим, что при таком задании ориентации границы $(m + 1)$-связной
области кривые $\Gamma_i$, $i = 1,...,m$, будут ориентированы отрицательно по
отношению к ограниченным областям $G_i$, кривые же $\Gamma_i^-$ , наоборот, будут
положительно ориентированы по отношению к $G_i$.

\subsection{Формула Грина для многосвязных областей:}
	Пусть область $G$ $(m + 1)$-связна, ее внешний и внутренние контуры $\Gamma_0$, $\Gamma_1,$...$, \Gamma_m$ являются замкнутыми кусочно-гладкими
кривыми без самопересечений, и пусть граница области $G$ положительно ориентирована. Далее, пусть $P(x, y)$ и $Q(x, y)$ есть заданные функции
такие, что

1)  $P$ и $Q$ непрерывны в замкнутой области $G$


2) $P$ и $Q$ имеют непрерывные частные производные $\frac{\partial Q}{\partial x},\frac{\partial P}{\partial y} $ в замкнутой $G$

Тогда имеет место равенство
\begin{equation}\label{eq3}
\int\limits_G \int (\frac{\partial Q}{\partial x} 
-
\frac{\partial P}{\partial y}) dxdy 
=
\int\limits_{\Gamma_0} (Pdx + Qdy)
-
\sum_{i = 1}^m \int\limits_{\Gamma_i^-} (Pdx + Qdy)
\end{equation}

\textit{Док-во для двусвязной области $G$:}

\begin{wrapfigure}{r}{8cm}
\includegraphics[width = 8cm]{многосвязная.jpg}
\end{wrapfigure}

Соединим область $G_1$ с кривой $\Gamma_0$ разрезом, который представляет собой кусочно-гладкую кривую без самопересечений. Обозначим разрез как $L$



Обозначим через $G^*$ область, полученную из $G$ удалением данного разреза, предполагая, что граница области $G^*$ состоит из границы $G$ (с сохранением ориентации) и разреза, проходимого дважды.Граница $G^*$ представляет собой кусочно-гладкую кривую, а значит по Формуле Грина для односвязной области имеем:

$$
\int\limits_{G^*} \int (\frac{\partial Q}{\partial x} 
-
\frac{\partial P}{\partial y}) dxdy 
=
\int\limits_{\delta G^*} (Pdx + Qdy)
$$	
Далее, поскольку двойной интеграл не меняется при присоединении к множеству интегрирования множества нулевой двумерной меры, то имеет место равенство
$$
\int\limits_{G^*} \int (\frac{\partial Q}{\partial x} 
-
\frac{\partial P}{\partial y}) dxdy 
=
\int\limits_{\delta G} (Pdx + Qdy)
= 
\int\limits_{G} \int (\frac{\partial Q}{\partial x} 
-
\frac{\partial P}{\partial y}) dxdy 
$$

Заметим, что $\delta G = \Gamma_0 \cup L \cup \Gamma_1$, вспоминая определение (\ref{eq1}) интеграла по кусочно-гладкой кривой:
$$
\int\limits_{\delta G} (Pdx + Qdy)
=
\int\limits_{\Gamma_0} (Pdx + Qdy)
+
\int\limits_{L^+} (Pdx + Qdy)
+
\int\limits_{L^-} (Pdx + Qdy)
+
\int\limits_{\Gamma_1} (Pdx + Qdy)
$$
Учитывая, что направление движения по кривой имеет значение:
$$
\int\limits_{G} \int (\frac{\partial Q}{\partial x} 
-
\frac{\partial P}{\partial y}) dxdy 
=
\int\limits_{\delta G} (Pdx + Qdy)
=
\int\limits_{\Gamma_0} (Pdx + Qdy)
-
\int\limits_{\Gamma_1^-} (Pdx + Qdy)
$$
Таким образом, мы получили формулу (\ref{eq3}) для случая двусвязной области $G$

Что будет в случае, если $G$ - $(m+1)$-связная область?
Да то же самое, только 
$\delta G = \Gamma_0 \cup (L_1 \cup$ ... $ \cup L_m)\cup (\Gamma_1 \cup$ ... $\cup\Gamma_m)$
$$
\int\limits_{\delta G} (Pdx + Qdy)
=
\int\limits_{\Gamma_0} (Pdx + Qdy)
+
\sum_{i = 1}^m (\int\limits_{L_i^+}(Pdx + Qdy) - \int\limits_{L_i^-} (Pdx + Qdy))
-
\sum_{i = 1}^m \int\limits_{\Gamma_i^-} (Pdx + Qdy)
$$
Отсюда немедленно получаем
$$
\int\limits_{\delta G} (Pdx + Qdy)
=
\int\limits_{\Gamma_0} (Pdx + Qdy)
-
\sum_{i = 1}^m \int\limits_{\Gamma_i^-} (Pdx + Qdy)
$$
читд

\subsection{Теорема 1.2}

Пусть

1)  $P$ и $Q$ непрерывны в замкнутой, связной области $G$


2) $P$ и $Q$ имеют непрерывные частные производные $\frac{\partial Q}{\partial x},\frac{\partial P}{\partial y} $ в замкнутой $G$

тогда 4 свойства эквивалентны:


1) Независимость $P(x,y)$, $Q(x,y)$ от пути интегрирования в $G$

2) Для любой замкнутой кусочно-гладкой кривой $\Gamma$, целиком лежащей в $G$, выполняется
$$
\int\limits_{\Gamma} (Pdx + Qdy) = 0
$$

3) Существует функция $u(x, y$) такая, что для любых 
точек $(x, y)$ из $G$ выполняется

$$
du(x, y) = P(x, y) dx + Q(x, y) dy;
$$

4) Для любых точек $(x, y)$ из $G$ выполняется
$$
\frac{\partial Q(x,y)}{\partial x} = \frac{\partial P(x,y)}{\partial y} 
$$

\newpage
\textit{Док-во:} 
$(1 \Rightarrow 2)$
$$
\int\limits_{\Gamma_1} (Pdx + Qdy) 
=
\int\limits_{\Gamma_2} (Pdx + Qdy) 
$$
$$
\int\limits_{\Gamma_1} (Pdx + Qdy) 
-
\int\limits_{\Gamma_2} (Pdx + Qdy) 
=
0
$$
$$
\int\limits_{\Gamma_1} (Pdx + Qdy) 
+
\int\limits_{\Gamma_2^-} (Pdx + Qdy) 
=
0
$$
\includegraphics[width = 8cm]{4экв.jpg}	

В силу того, что $\Gamma = \Gamma_1 \cup \Gamma_2^-$(с учетом направления):
$$
\int\limits_{\Gamma} (Pdx + Qdy) 
=
\int\limits_{\Gamma_1} (Pdx + Qdy) 
+
\int\limits_{\Gamma_2^-} (Pdx + Qdy) 
=
0
$$

В силу произвольности выбора $\Gamma_1$ и $\Gamma_2$ получаем, что $\Gamma$ - тоже произвольная кривая.


$(2 \Rightarrow 1)$

	Пусть $\Gamma$ - замкнутая кусочно-гладкая кривая
	и выполняется $\int\limits_{\Gamma} (Pdx + Qdy) = 0$
	
Разобьем(с учетом направления) $\Gamma = \Gamma_1 \cup \Gamma_2$, где $\Gamma_1$ и $\Gamma_2$ - кусочно-гладкие или просто гладкие кривые.
Тогда:
$$
\int\limits_{\Gamma} (Pdx + Qdy) 
=
\int\limits_{\Gamma_1} (Pdx + Qdy) 
+
\int\limits_{\Gamma_2} (Pdx + Qdy) 
=
0
$$
Отсюда:
$$
\int\limits_{\Gamma_1} (Pdx + Qdy) 
=
\int\limits_{\Gamma_2^-} (Pdx + Qdy) 
$$
\newpage
$(1 \Rightarrow 3)$\textbf{Надо еще исправлять}
Пусть $M_0 = (x_0, y_0)$ есть фиксированная точка $G$, $M = (x^*, y^*)$ есть текущая точка $G$, $\Gamma : M_0M$ есть кусочно-гладкая кривая без самопересечений,
целиком лежащая в $G$ и соединяющая точки $M_0$ и $M$.
Пусть 
$$
u(x,y) = \int\limits_{M_0M} (Pdx + Qdy) 
=
\int\limits_{\Gamma} (Pdx + Qdy) 
$$
В силу условия \hyperref[eq4]{связности} $G$:
$ \exists h : (x^* + h , y^*) \in G$

Пусть прямая L, прямая соединяющая $M(x^* , y^*)$ и $M^*(x^* + h , y^*)$

Покажем, что 

$$u_x = \lim_{h \to 0}\frac{u(x^* + h , y^*) - u(x^* , y^*)}{h} = P(x^* , y^*) $$
\includegraphics[width = 13cm]{Т1.2.jpg}	

Имеем
$$
\frac{u(x^* + h , y^*) - u(x^* , y^*)}{h} 
= 
\frac{1}{h}(u(x^* + h , y^*) - u(x^* , y^*))
$$
$$
=
\frac{1}{h} \int\limits_{M}^{M^*} (Pdx + Qdy)
=
\frac{1}{h} \int\limits_{L} (Pdx + Qdy)
$$
Параметризуем отрезок L:

	$x = x^* + th , t \in [0,1] \Rightarrow dt = dx$
	
	$y = y^* \Rightarrow dy = 0$
$$
\int\limits_{L} (P(x,y)dx + Q(x,y)dy) = \int\limits_{0}^{1}P(x^* + th,y^*)dt
=
P(x^* + h,y^*) - P(x^*,y^*)
$$	
Применим формулу \hyperref[eq5]{конечных приращений Лагранжа}
$$
P(x^* + h,y^*) - P(x^*,y^*) = P(x^* + \theta h,y^*)h  ,\theta \in (0,1)
$$
Отсюда:
	
$$
P(x^* + \theta h,y^*) 
=
\frac{u(x^* + h , y^*) - u(x^* , y^*)}{h}
$$
Теперь при $h \to 0$	 получаем:

$$
P(x^* , y^*) 
=
u_x
$$
Аналогично доказываем, что $Q(x^* , y^*) = u_y$

$$
du(x^* , y^*) = u_x dx + u_y dy = P(x^* , y^*) dx + Q(x^* , y^*) dy
$$
Но нам нужно еще доказать дифференцируемость $u(x,y)$ в $G$

$u_x = P(x , y)$
$u_y = Q(x , y)$

Тк по условию у нас $P$ и $Q$ имеют непрерывные частные производные $\frac{\partial Q}{\partial x},\frac{\partial P}{\partial y} $ в замкнутой $G$, то существую вторые производные для $u(x,y)$, отсюда немедленно следует дифференцируемость$u(x,y)$
читд	
	
$(3 \Rightarrow 1)$	

\textbf{Одно звено} : $M_0(x_0 , y_0)$ и $M(x^* , y^*)$ 

\includegraphics[width = 13cm]{1.jpg}
$$
\int\limits_{\Gamma} (Pdx + Qdy) 
=
\int\limits_{\Gamma} (u_xdx + u_ydy) 
$$

Параметризуем $\Gamma$:

 $ x = \varphi(t)$
 
 $ y = \psi(t)$ , где $t \in [\alpha , \beta]$
$$
\int\limits_{\Gamma} (Pdx + Qdy)
=шо
\int\limits_{\alpha}^{\beta} P(\varphi'(t) , \psi'(t))*\varphi'(t)dt 
+ Q(\varphi'(t) , \psi'(t))*\psi'(t)dt 
=
$$
$$
\int\limits_{\alpha}^{\beta} \frac{d}{dt}(u(\varphi(t) ,\psi(t)))dt 
= 
u(\varphi(\beta) ,\psi(\beta)) -  u(\varphi(\alpha) ,\psi(\alpha))
$$
Получается, что интеграл зависит лишь от начальных точек, а значит не зависит от пути интегрирования

Если \textbf{n звеньев:}

\includegraphics[width = 13cm]{n.jpg}	

$$
u(x_1 , y_1) - u(x_0 , y_0) 
+ 
u(x_2 , y_2) - u(x_1 , y_1)
+ 
u(x_3 , y_3) - u(x_2 , y_2)
+ ... +
u(x^* , y^*) - u(x_{n-1} , y_{n-1})
=
$$
$$
=
u(x^* , y^*) - u(x_0 , y_0)
$$
Получается, что и от количества звеньев не зависит

читд

$(3 \Rightarrow 4)$
$$
u_{xy}(x,y) = \frac{\partial P}{\partial y} = P_y
$$
$$
u_{yx}(x,y) = \frac{\partial Q}{\partial x} = Q_x
$$
В силу непрерывности $P_y , Q_y$, получается, что и
$u_{xy}(x,y), u_{yx}(x,y)$ непрерывны в $G$. А если если существуют смешанные непрерывные производные, то они равны.

\hyperref[eq6]{Теорема о смешанных производных}
$$
u_{xy} = u_{yx} \Rightarrow Q_x = P_y
$$

читд

$(4 \Rightarrow 2  \Rightarrow 3)$
	Пусть $P_y = P_y$

Рассмотрим кусочно-гладкую замкнутую кривую в замкнутой $G$

Тогда справедлива формула Грина:
$$
\int\limits_{\Gamma} (Pdx + Qdy)
=
\int\limits_{G} \int (\frac{\partial Q}{\partial x} 
- 
\frac{\partial P}{\partial y}) dxdy
=
0
$$
Отсюда получаем свойство 2:
$$
\int\limits_{\Gamma} (Pdx + Qdy)
=
0
$$
Ранее уже доказали, что $(2 \Rightarrow 3)$

читд

\section{Поверхности в $R^n$}
\subsection*{Определение 1.13 }

Пусть $G$ есть ограниченная область из пространства $R^2$,
$f(u, v),g(u, v), h(u, v)$ — определенные при $(u, v) \in G$ и непрерывные
на $G$ функции. \textbf{Непрерывной поверхностью $S$} называется множество:

$
S = \{(x, y, z) : x = f(u, v), y = g(u, v), z = h(u, v), (u, v) \in G \}
$

Вектор-функция $\Phi(u, v)= (f(u, v), g(u, v), h(u, v))$ называется
представлением, или \textbf{параметризацией} поверхности.	
	
\subsection*{Определение 1.14 }
	
Рассмотрим точку $(u_0 , v_0) \in S$
$$
(u_0 , v_0) = \begin{cases}
	\textbf{ не особая}, \text{если }  \Phi_u(u_0 , v_0), \Phi_v(u_0 , v_0) - \text{ЛН};\\
	\textbf{ особая}, \text{если } \Phi_u(u_0 , v_0) , \Phi_v(u_0 , v_0) - \text{ЛЗ};
\end{cases}
$$

\subsection*{Определение 1.15 }

Поверхность называется \textbf{гладкой}, если все ее точки не особые.



\subsection*{Определение 1.16 }
Совокупность каcательных прямых к поверхности в точке -\textbf{касательная плоскость} к поверхности в этой точке.

\subsection*{Определение 1.17}
	
	Пусть задана поверхность $S$ и $M_0(x_0 , y_0 , x_0) \in S,  
$ где

$
x_0 = f(u_0 , v_0) , y_0 = g(u_0 , v_0) , 
z_0 = h(u_0 , v_0)$

Тогда \textbf{касательная к плоскости $S$ в $(u_0 , v_0)$} определяется через определитель данной матрицы:
\begin{center}
\begin{equation}
\includegraphics[width = 13cm]{kasatelnya.jpg}\label{eq7}
\end{equation}
\end{center}
\subsection*{Определение 1.18}
	Прямая, проходящая через точку касания поверхности с касательной плоскостью и перпендикулярная этой плоскости, называется \textbf{нормальной прямой к поверхности в указанной точке.}
	Нормаль определяется матрицей:
\begin{equation}
	\includegraphics[width = 13cm]{normal.jpg}\label{eq8}
\end{equation}
\subsection*{Замечание}
	У плоскости в точке есть две нормали, верная из них та, которая задается матрицей из определениия 1.17(\ref{eq8}).	
	
	
\textbf{Важный факт}
	В каждой точке гладкой поверхности S однозначно определена нормаль,вычисляемая по формуле (\ref{eq8}).
	
\subsection*{Определение 1.19}
Если на поверхности $S$ эта нормаль
меняется непрерывно, то поверхность $S$ называется \textbf{ориентированной}. При задании ориентации поверхности считается, что поверхность S является \textbf{ двусторонней }, и та сторона поверхности,
которая прилегает к нормали (\ref{eq8}), называется 
\textbf{ положительной стороной и обозначается $S^+$ }, противоположная же сторона называется 
\textbf{ отрицательной и обозначается $S^-$ }	.

Пример неориентированной поверхности: Лента Мебиуса
	

\subsection*{Определение 1.20}

Две поверхности $S_i$ и $S_j$ называются \textbf{соседними}, если кривые
$\Gamma_i$ и $\Gamma_j$ имеют одну или несколько общих дуг (общих участков,не вырождающихся в точку).

\subsection*{Определение 1.21 }
Поверхность называется \textbf{кусочно-гладкой}, если она состоит конечного числа гладких поверхностей, которые могут пересекаться лишь по своим граничным точкам.

Если $S$ - кусочно-гладкая поверхность, то ее можно представить в виде:

	$S = S_1 \cup S_2 \cup ... \cup S_p$, где $S_i$ и $S_j$ или соседние или могут быть соединены некоторой последовательностью поверхностей.
	
	Пример: Кубик
	
\includegraphics[width = 8cm]{cube.jpg}

\subsection*{Определение 1.22 }
Кусочно-гладкая поверхность S, состоящая из $m$ частей
$S_1$,...,$S_m$, называется \textbf{ориентируемой}, если существует такая ориентация кривых $\Gamma_1$, ..., $\Gamma_m$ (границ поверхностей $S_1,...,S_m$), что
части (дуги) этих кривых, принадлежащие двум различным кривым
$\Gamma_i$ и $\Gamma_j$, получают от них противоположную ориентацию.

\subsection{I квадратичная форма поверхности}
	
	Введем обозначения:
		
	$E(u , v) = (\vec{\Phi}_u(u , v) , \vec{\Phi}_u( u, v)) = |\vec{\Phi}_u( u, v)|^2$ - квадрат модуля
	
	$G(u , v) = \vec{(\Phi}_v(u , v) , \vec{\Phi}_v( u, v)) = |\vec{\Phi}_v^2( u, v)|^2$ - квадрат модуля
	
	$F(u , v) = (\vec{\Phi}_u(u , v) , \vec{\Phi}_v( u, v))$ 
	
	Тогда:
	
	$$(d\vec{\Phi})^2 = (\vec{\Phi}_u du + \vec{\Phi}_v dv)^2 = Edu^2 + 2Fdudv + Gdv^2$$
	
	I квадратичная форма имеет вид:
	$$Edu^2 + 2Fdudv + Gdv^2$$			
	
\textbf{Замечание:}
	I квадратичная форма не отрицательна
	
\textit{Док-во:}
Как известно из курса Линала, квадрат скалярного произведения неотрицателен.
	$$Edu^2 + 2Fdudv + Gdv^2 = (d\vec{\Phi})^2 \geqslant 0$$
	
\subsection{Поверхностный интеграл I рода}
	Пусть $S : \Phi(u , v)$ - Гладкая поверхность и задана функция $\psi(u , v)$, тогда \textbf{поверхностным интегралом I рода от $\psi(u,v)$} назовем:
	
$$
\int\limits_{S} \psi(u , v) ds 
=
\int\limits_{\Omega}\int \psi(u,v)\sqrt{EG - F^2}dudv	
$$		
	
	Также введем меру:
$$
	\int\limits_{S}ds = mesS;
$$	
	
\subsection{Свойства поверхностных интегралов I рода:}
	1) Линейность
$$
\int\limits_{S} (\alpha F_1 + \beta F_2)ds
=
\alpha\int\limits_{S}  F_1ds
+
\beta\int\limits_{S}  F_2ds
$$
	2)Аддитивность	
$$
\int\limits_{S}  F_2ds
+
\int\limits_{S}  F_1ds
$$



\subsection{Поверхностный интеграл II рода}
\subsection*{Некоторые факты про нормаль}

	Пусть есть поверхность $S$ с заданной параметризацией $\Phi(u,v)$ , где $(u,v)$ из замкнутой области $\Omega$. Если $\vec{n}$ - нормаль, вычисляемая по (\ref{eq8}), то единичная нормаль будет вычисляться так:
	$$\vec{V} = \frac{\vec{n}}{|n|}$$

\begin{wrapfigure}{r}{7.5cm}
\includegraphics[width = 7.5cm]{косинусы.jpg}
\end{wrapfigure}	
	
Однако единичную нормаль можно задать по-другому:

Пусть $\vec{n} = (n_x , n_y , n_z)$

Тогда 
$$
n_x = |n|cos\alpha;
$$
$$
n_y = |n|cos\beta;
$$
$$
n_z = |n|cos\gamma;
$$
...

Тогда вектор 
$$
(cos\alpha , cos\beta , cos\gamma) = (\frac{n_x}{|n|} , \frac{n_y}{|n|} , \frac{n_z}{|n|})
$$
А это не что иное как нормированный вектор $\vec{n}$, те $\vec{V}$

\subsection*{Определение 1.23:}\label{eq9}
\textbf{	Поверхностным интегралом II рода по поверхности $S$ по переменным $x,y$ назовем}
$$
\int\limits_{S}\int \Psi dxdy 
=
\int\limits_{S^+}\int \Psi dxdy 
=
\int\limits_{S} \Psi cos\gamma ds 
$$
	
Соотвествено, если интеграл по $y , z$ будет такой же, но с $cos\alpha$

\textbf{Замечание:}
$$
\int\limits_{S^+}\int \Psi dxdy 
=
-\int\limits_{S^-}\int \Psi dxdy
$$

\subsection{Свойства поверхностных интегралов II рода}
	1) Линейность
	
	2) Аддитивность
	
	3) Зависимость от стороны поверхность(замечание выше)

\subsection*{Определение 1.24}
	Пусть $S$ есть кусочно-гладкая поверхность, состоящая из частей $S_1,...,
S_m$, и пусть на $S$ имеется согласованная ориентация $S^+$. Далее, пусть на $S$
задана функция $F(x, y, z)$. \textbf{Поверхностным интегралом второго рода по поверхности $S^+$ по координатам $x, y$ называется сумма}:

$$
I
=
\sum_{i = 1}^{m} \int\limits_{S_i^+}\int F dxdy 
$$

Для $dxdz , dydz$ определяется аналогично. 


\newpage
\subsection{Формула Гаусса-Остроградского:}
\textbf{Обозначаю: $R = R(x, y ,z) , Q = Q(x,y,z) , P = P(x,y,z)$;}

		Пусть $V$ есть ограниченная поверхностью $\Omega$, область из пространства $R^3$, и $V$ можно разбить кусочно-гладкими поверхностями на конечное число элементарных областей. Далее, пусть $P(x, y ,z)$, $Q(x, y ,z)$ и $R(x,y,z)$ есть заданные функции
такие, что 

1)  $P$, $Q$ и $R$ непрерывны в замкнутой области $V$


2) $P$,$Q$ и $R$ имеют непрерывные частные производные $\frac{\partial Q}{\partial x},\frac{\partial P}{\partial y},
\frac{\partial R}{\partial z} $ в замкнутой $V$

Тогда верная формула Гаусса-Остроградского:
$$
\int\int\limits_{V}\int (\frac{\partial P}{\partial x} + \frac{\partial Q}{\partial y} + \frac{\partial R}{\partial z})dxdydz
=
\int\limits_{S} (Pcos\alpha + Qcos\beta + Rcos\gamma)ds
$$
в которой $(cos\alpha, cos\beta, cos\gamma)$ есть направляющие косинусы вектора \textbf{внешней} нормали к границе $S$ области $V$

\includegraphics[width = 12cm]{Гаусс-Остроградский.jpg}

\textit{Док-во для элементарной $V$:}

	Поверхность $S$ можно представить как:
	$S = S_0 \cup S_1^- \cup S_2^+$

	Коль $V$ элементарна, значит и по $Oz$ элементарна, а значит
	$\varphi(x,y) \leq z \leq \psi(x,y)$
	


	Рассмотрим 
	$$\int\int\limits_{V}\int R_z(x,y,z)$$



Тк $V$ элементарна по $Oz$: 
$$
\int\int\limits_{V}\int R_z(x,y,z) dxdxydz
=
\int\limits_{\Omega}\int(\int_{\varphi(x,y)}^{\psi(x,y)} R_z(x,y,z)dz) dxdy
=
$$
$$
=
\int\limits_{\Omega}\int (R(x,y,\psi(x,y)) - R(x,y,\varphi(x,y)))dxdy
$$

Тк 

1)	$\varphi(x,y) \leq z \leq \psi(x,y)$ 

2) $\vec{n_1}$ - внешний к $z = \psi(x,y)$ , но $\vec{n_2}$ - внутренний к $z = \varphi(x,y)$ 

3) \hyperref[eq10]{$cos\gamma_1 = \frac{1}{\sqrt{1 + \psi_x^2 + \psi_y^2 }}$} для $S_2$ , и $cos\gamma_2 = \frac{1}{\sqrt{1 + \varphi_x^2 + \varphi_y^2 }} $ для $S_1$

4)$\psi(x,y) : \sqrt{EG - F^2} = \sqrt{(1 + \psi_x^2)(1 + \psi_y^2) - (\psi_x \psi_y)^2} = \sqrt{1 + \psi_x^2 + \psi_y^2 }$

  \par $\varphi(x,y) : \sqrt{EG - F^2} = \sqrt{(1 + \varphi_x^2)(1 + \varphi_y^2) - (\varphi_x \varphi_y)^2} = \sqrt{1 + \varphi_x^2 + \varphi_y^2 }$

Можно сделать вывод:
$$
\int\limits_{\Omega}\int R(x,y,\psi(x,y))dxdy 
\stackrel{?}{=}
\int\limits_{S_2^+}\int R(x,y,z)  cos\gamma_1  \sqrt{EG - F^2} dxdy
=
\int\limits_{S_2^+}\int R(x,y,z)dxdy
$$
$$
\int\limits_{\Omega}\int R(x,y,\varphi(x,y))dxdy 
\stackrel{?}{=}
\int\limits_{S_2^+}\int R(x,y,z)  cos\gamma_2  \sqrt{EG - F^2} dxdy
= 
\int\limits_{S_1^+}\int R(x,y,z)dxdy
$$


По определению \hyperref[eq9]{поверхностного интеграла II рода}, учитывая то, что для $S_2^+$
нормаль будет внешней, а для $S_1^+$ - внутренней, можно переписать наши поверхностные интегралы II рода,
как поверхностные интегралы I рода:
	$$\int\limits_{S_2^+}\int R(x,y,z)dxdy = \int\limits_{S_2} R(x,y,z)cos\gamma_1 ds$$
	$$\int\limits_{S_1^+}\int R(x,y,z)dxdy = -\int\limits_{S_1} R(x,y,z)cos\gamma_2 ds$$

Тогда, зная, что $S_0$ - боковая поверхность(это интеграл равен нулю, тк $\gamma_3 = 90$(нормаль к $S_0$ перпендикулярная $Oz$),а значит $cos\gamma_3 = 0$ ), Пусть $\gamma$ - угол между номалью к поверхности $S$ и положительным направлением оси $Oz$ ,
тогда можно сделать вывод:
$$
\int\int\limits_{V}\int \frac{\partial R}{\partial z}dxdydz = \int\limits_{S_2^+} R(x,y,z)cos\gamma_1 ds + 
\int\limits_{S_0} R(x,y,z)cos\gamma_3 ds - \int\limits_{S_1^+} R(x,y,z)cos\gamma_2 ds =
$$
$$
=
\int\limits_{S} R(x,y,z)cos\gamma ds 
$$

(\hyperref[eq11]{\textbf{Замечание}})

Аналогично доказывается:
$$
\int\int\limits_{V}\int \frac{\partial P}{\partial x}dxdydz = \int\limits_{S} R(x,y,z)cos\alpha ds 
$$
$$
\int\int\limits_{V}\int \frac{\partial Q}{\partial y}dxdydz = \int\limits_{S} R(x,y,z)cos\beta ds 
$$

Суммируя, получаем нужную формулу.

читд

\textit{Док-во для V - составленной из гладких поверхностей:}

	Пусть теперь область $V$ есть множество $V = V_1 \cap V_2 \cap S^* $, причем $V_1$
и $V_2$ есть элементарные области, $S^* $ есть разделяющая их кусочно-гладкая
поверхность. Представляя интеграл по области $V$ в виде суммы интегралов по областям $V_1$ и $V_2$ (что возможно вследствие свойства аддитивности
тройного интеграла), применяя формулу \textbf{Гаусса-Остроградского для элементарной области} к каждой области $V_1$ и $V_2$,
учитывая, что внешняя нормаль на поверхности $S^* $ направлена взаимно
противоположно по отношению к областям $V_1$ и $V_2$, а также то, что оставшиеся части границ областей $V_1$ и $V_2$ составят вместе границу $V$, получим
требуемую формулу  для составной области $V$.

	Если область G составлена из более чем двух областей
$V_1$ и $V_2$ и разделяющих их поверхностей, то рассуждения будут вполне
аналогичны, и тем самым формула \textbf{Гаусса-Остроградского для элементарной области} будет справедлива и для такой
области.


\newpage
\subsection{Формула Стокса:}

\noindent 1) Пусть $S$ - поверхность в $R^3$, заданная своей вектор-функцией $\Phi(u,v)$ , $(u,v) \in \overline{\Omega}$

\noindent 2) Пусть вектор-функция $\Phi(u,v)$ есть дважды непрерывно дифференцируемая при \par $(u,v) \in \overline{\Omega}$ функция
	
\noindent 3) Пусть $\Omega$ есть плоская ограниченная область такая, что для нее выполняется \par \hyperref[eq3]{формула Грина}

\noindent 4) Путь $\gamma_0$ - граница $\Omega$ , $\gamma_0$ - замкнутая, кусочно-гладкая без самопересечений с \par положительным направление обхода с параметризацией  $\gamma_0 : u(t) , v(t) , t \in [a,b]$

\noindent 5)На $S$ определена нормаль $\vec{V}(cos\alpha , cos\beta , cos\gamma)$

\noindent 6)Определим кривую $\gamma$ в пространстве $R^3$ как кривую с параметризацией \par $\Phi(u(t) , v(t)) , t \in [a,b], $ и пусть эта кривая представляет собой границу,или край \par поверхности $S$ (говорят также, что поверхность $S$ натянута
на кривую $\gamma$).

\noindent 7) Пусть область $G$ из пространства $R^3$ есть такая область, что
выполняется\par  вложение $S \subset G$, и пусть функции $P(x, y, z), Q(x, y, z),
R(x, y, z)$ определены при \par $(x, y, z) \in G$

\textbf{Формула Cтокса}

1)Пусть функции $P(x, y, z), Q(x, y, z), R(x, y, z)$ непрерывны в области $G$ 

2) Все частные производные $P , Q , R$ тоже непрерывны в $G$ (9 штук)

3)Для поверхности $S$, для кривых $\gamma_0$ и $\gamma$ выполняются сделанные выше
предположения. Тогда выполняется равенство: 

\begin{equation}\label{eq11}
\int\limits_{\gamma} Pdx + Qdy + Rdz
=
\int\limits_{\gamma} [(R_y - Q_z)cos\alpha + (P_z - R_x)cos\beta	+ (Q_x - P_y)cos\gamma]ds
\end{equation}

\textit{Док-во}
Потом



\newpage
\section{Элементы векторного анализа}
	Будем работать в пространстве $R^3$. $G$ область из $R^3$ и $P(x,y,z), Q(x,y,z), R(x,y,z)$ - заданные функции, определенные при 
	$(x,y,z) \in G$
	\textbf{Обозначаю: $R = R(x, y ,z) , Q = Q(x,y,z) , P = P(x,y,z)$;}
	
\subsection*{Определение 1.25}
	\textbf{Векторным полем} назовем совокупность векторов 
	
	$\vec{a}(x,y,z) = (P(x,y,z) , Q(x,y,z), R(x,y,z))$, где $(x,y,z) \in G$
	
\subsection*{Определение 1.26}
Оператор \textbf{"наблa":}
	
	

	$$\nabla = ( \frac{\partial}{\partial x} ,  \frac{\partial}{\partial y} , \frac{\partial}{\partial z})$$
	
\textbf{Градиент}
	Тогда, если $f$ - функция, то  
	$$\nabla f =   \frac{\partial f}{\partial x} \vec{i} +  \frac{\partial f}{\partial y}\vec{j} + \frac{\partial f}{\partial z} \vec{k}= grad f \text{ - вектор}$$ 
 	Если есть вектор $\vec{F} = (P, Q, R)$, то
 		$$\nabla \vec{F} = \vec{i}  \frac{P}{\partial x} + \vec{j} \frac{Q}{\partial y} + \vec{k}\frac{R}{\partial z} \text{ - вектор}$$ 
	
\textbf{Дивергенция:}
	Скалярное произведение $\nabla $ и  $\vec{F}= (P , Q , R)$:
	
	$$(\nabla , \vec{F}) = div \vec{F} = \frac{\partial P}{\partial x} + \frac{\partial Q}{\partial y} + \frac{\partial R}{\partial z} \text{ - скаляр}$$ 

\textbf{Ротор:}
	Векторное произведение $\nabla $ и  $\vec{F}= (P , Q , R)$:

$$
[\nabla ,\vec{F}] = 
\begin{vmatrix}
\vec{i} & \vec{j} & \vec{k} \\
\frac{\partial}{\partial x} & \frac{\partial}{\partial y}& \frac{\partial}{\partial z} \\
P & Q & R
\end{vmatrix}
$$

\textbf{Циркуляция:}
	Пусть $\gamma$ есть замкнутая кусочно-гладкая кривая, без самопересечений,
лежащая в $G$. Интеграл второго рода 
	$$\int\limits_{\gamma} P dx + Q dy + R dz$$
называется циркуляцией векторного поля $\vec{a}(x, y, z)$ по кривой $\gamma$, \textbf{если интеграл существует}

\textbf{Поток векторного поля:}

	Пусть $S$ есть некоторая ориентированная поверхность, лежащая в $G$,
и пусть ее ориентацию определяет единичная нормаль 
$V$, вычисляемая с помощью формулы (\ref{eq8}), в случае если поверхность $S$ не является границей некоторой области $G'$
, лежащей в $G$, или же внешняя нормаль, если
поверхность $S$ является границей области $G'$(просто должен получиться \textit{ежик}, чтобы все нормали не были направлены внутрь).
 
\textbf{Интеграл первого рода}
	$$\int\limits_{S} (\vec{a} , \vec{v})ds$$
\textbf{($(\vec{a} , \vec{v})$ — скалярное произведение векторов $\vec{a}$ и $\vec{v}$) называется потоком векторного поля $\vec{a}(x, y, z)$ через поверхность $S$.}

\subsection{Замечание:}
	Можно переформулировать формулы Гаусса-Остроградского и Стокса:
	
	Пусть выполняются все условия теоремы про формулу Гаусса-Остроградского . Тогда интеграл от дивергенции векторного поля $\vec{a}(x, y, z)$ по области $G$ равен потоку этого поля через границу $G$.


	Пусть выполняются все условия теоремы про формулу Стокса. Тогда циркуляция векторного поля $\vec{a}(x, y, z)$ по контуру $\gamma$ равна потоку ротора этого поля через поверхность $S$, натянутую на $\gamma$.
	
\subsection{Определение 1.27}
Пусть $G$ есть некоторая область из пространства $R^3$, и пусть в $G$ задано
векторное поле $\vec{a}(x, y, z).$
Если циркуляция векторного поля $\vec{a}(x, y, z)$ по любой замкнутой кусочно-гладкой кривой без самопересечений, лежащей в области $G$, равна нулю, то это поле называется \textbf{потенциальным}.

\subsection{Замечание}
	Векторное поле потенциальное $\Leftrightarrow$ оно является безвихревым(при условии существования$ P, Q, R$)
	
	
\subsection{Определение 1.28}	
	Поверхность, ограничивающая некоторую область - \textbf{допустима}, если к ней можно применить формулу Гаусса-Остроградского.
	
\subsection{Определение 1.29}
	$G$ - \textbf{объемно-односвязная}, если для любой замкнутой допустимой поверхности $S$, ее внутренность лежит в G(трехмерная область без дырок).
	
\subsection{Определение 1.30}
Заданное в области G векторное поле $a(x, y, z)$ называется \textbf{соленоидальным}, если его \textbf{поток} через любую лежащую в $G$ \textbf{допустимую} поверхность равен нулю.	
	
\subsection{Теорема	Гельмгольца}
	\par \textbf{Формулировка с лекции:}
	
		Непрерывно дифференциемое векторное поле соленоидально в объемно-односвязной $G$ $\Leftrightarrow$ дивергенция в каждой точке 		равна нулю.
	
	\par \textbf{Оригинальная формулировка(хз зачем):}
	
		Любое векторное поле $F$ , однозначное, непрерывное и ограниченное во всем пространстве, может быть разложено на сумму потенциального и соленоидального векторных полей	
	
	\textit{б/д}
	
	
\part{Элементы функционального анализа}
\section{Метрические пространства}
\subsection*{Определение 2.1:}
	\textbf{Метрическое пространство:}
	
	Пусть задано множество $X$.Тогда отображение $\rho: X \times X \to R_+$ - метрика, если 
	
	$\forall x,y,z \in X$
	
	1) $\rho(x,y) \ge 0$ , $\rho(x,y) = 0 \Leftrightarrow x = y$
	
	2) $\rho(x,y) = \rho(y,x)$
	
	3) $\rho(x,y) = \rho(x,z) + \rho(z,y) $
	
\textbf{Пример:}
	$$
P(x,y) = \begin{cases}
   0, &\text{если } x = y; \\
   1, &\text{если } x \neq y;
\end{cases}
$$	
	
\subsection*{Определение 2.2:}
	\textbf{Открытый шар с радиусом $R$:}
	$B_R(x_0) = \{{x \in X : \rho(x,x_0) < R}\} , x_0 \in X$
	
\subsection*{Определение 2.3:}
	\textbf{Замкнутый шар с радиусом $R$:}
	$B_{\bar{R}}(x_0) = \{{x \in X : \rho(x,x_0) \leq R}\} , x_0 \in X$

\subsection*{Определение 2.4:}
	\textbf{Сфера с радиусом $R$:}
	$S_R(x_0) = \{{x \in X : \rho(x,x_0) = R}\} , x_0 \in X$

\newpage
\subsection*{Определение 2.5:}
	\textbf{Предел последовательности точек из $M$:}
	
	Пусть задано метрическое пространство $(X,\rho)$ и $\{{x_n}\}_{n = 1}^{\infty}$ - последовательность элементов из $M$
	
	Тогда число $a$ - \textbf{предел данной последовательности}, если $\lim_{n \to \infty} \rho(x_n , a) = 0$
	
	
\textbf{Упражнение:}
	$x_0$ - предельная точка для $M$ $\Leftrightarrow$ $x_0$ - предел последовательности точек из $M$.
	
\section{Точки и множества из метрического пространства}	

\subsection*{Определение 2.6:}
	 Точка $x_0 \in X$ - \textbf{точка прикосновения для множества $M \subset X$}, если $\forall B_R (x_0)$ - содержит элементы множества $M$
	    
\subsection*{Определение 2.7:}
	 Точка $x_0 \in X$ - \textbf{предельная точка для множества $M \subset X$}, если $\forall B_R (x_0)$ - содержит элементы множества $M$,
	 не равные $x_0$
	
\subsection*{Определение 2.8:}
	 Точка $x_0 \in X$ - \textbf{внутренняя точка для множества $M \subset X$}, если $\exists B_R (x_0) \subset M$
	
\subsection*{Определение 2.9:}
	 Точка $x_0 \in X$ - \textbf{изолированная точка для множества $M \subset X$}, если $\exists B_R (x_0) \subset M$ такой, что
	 он не содержит точек из $M$, не равных $x_0$

\subsection*{Определение 2.10:}
	 Точка $x_0 \in X$ - \textbf{граничная точка для множества $M \subset X$}, если $\forall B_R (x_0) \subset M$ 
	 верно, что он содержит точки из $M$ и не из $M$

\subsection*{Определение 2.11:}
	 \textbf{Замыкание множества} - это присоединение ко множеству всех его предельных точек.

\subsection*{Определение 2.12:}
	 $\{{x_n}\}_{n = 1}^{\infty}$ - \textbf{фундоментальная последовательность в метрическом пространстве $X$}, если
	 
	 $\forall \varepsilon > 0$ $ \exists N = N(\varepsilon)$: 
	 если $\forall n > N, \forall m \ge 0$, то 
	 $\rho(x_{n+m}, x_n) < \varepsilon$
	 
\subsection*{Определение 2.13:} \label{eq101}
	 Множество $M \subset X$ \textbf{замкнуто}, если оно содержит все свои предельные точки.

\subsection*{Определение 2.14:}
	 Множество $M \subset X$\textbf{ совершенно}, если оно замкнуто и содержит все свои предельные точки.
	 
\subsection*{Определение 2.15:}
	 Множество $M \subset X$ \textbf{открытое}, если все его точки внутренние.
	 
\subsection*{Определение 2.16:}
	 Метрическое пространство $X$ - \textbf{полное}, если для любой фундоментальной последовательности его элементов, 
	 в $X$ найдется $x_0$: $\lim_{n \to \infty} = x_0$

\subsection*{Определение 2.17:}
 Множество $M \subset X$ \textbf{всюду плотное в $X$}, если его замыкание совпадает со всем $X$.
 
\subsection*{Определение 2.18:}
 Множество $M \subset X$ \textbf{нигде не  плотное в $X$}, если любой открытый шар пространства $X$
содержит открытый шар, свободный от точек множества $M$ .
	 	
\subsection*{Определение 2.19:}	 	
	 Метрическое пространство $X$ называется \textbf{сепарабельным,}
		если оно имеет счетное всюду плотное подмножество

\subsection*{Определение 2.20:}	 		
	Множество M метрического пространства X называется \textbf{связным}, если его нельзя представить виде объединения двух непустых отделимых множеств $M_1$ и $M_2$ .
		
\subsection*{Определение 2.21:}
	\textbf{Расстоянием} между непустыми подмножествами $M_1, M_2$ метрического пространства $X$ назовем:
	
	$inf \rho(x,y)$, где $x \in M_1 , y \in M_2$
	
\subsection*{Определение 2.22:}
	Множество $M \subset X$ - \textbf{область}, если оно открыто и связно.


\section{Линейные(векторные) пространства}
	Будем рассматривать линейные пространства над $R$
	
\subsection*{Определение 2.23:}
	$M$ - линейное(векторное) пространство, если выполняются следующие аксиомы:
	
	$\forall x,y,z \in M$  и любых скаляров $\lambda,\mu \in R_+$
	
	1)$ x + y \in M$
	
	2)$ \lambda x \in M$
	
	3)$x + y = y + x$
	
	4)$x + (y + z) = (x + y) + z$
	
	5)$\exists \Theta \in M : x + \Theta = \Theta + x = x$, где $\Theta$ - нуль пространства
	
	6)$(\lambda + \mu)x = \lambda x + \mu x$
	
	7)$\exists 1 \in M: 1*x = x * 1 = x$
	
	8)$0x = \Theta$

\textbf{Замечание:}
	Нуль пространства не всегда равен 0.
	
	Пример: Пространство матриц $m \times n$ имеет $\Theta$ - нулевая матрица, которая не равна 0.
	
\subsection*{Определение 2.24:}
	Отрезок в линейном пространстве: $[x,y] = \{ z : x = x + t(y-x) \}$, $t \in [0,1]$
\subsection*{Определение 2.25:}
	Линейное пространство \textbf{конечномерное}, если существует конечное множество его элементов, которое является ЛН и любой другой
	элемент из этого пространства выражается через их линейную комбинацию.
	

	
\subsection*{Определение 2.26:}
	Линейное пространство бесконечномерное, если для любого $n \in N$  в нем найдется система из $n$ независимых элементов.
	
\subsection*{Определение 2.27:}
 	$M$ - линейное многообразие, если 	
	
$$
\begin{cases}
   x,y \in M \Rightarrow \exists (x+y) \in M; \\
   x \in M , \lambda \in R \Rightarrow \exists \lambda x \in M.
\end{cases}
$$

\subsection*{Определение 2.28}
	Mножество $M \subset X$ - \textbf{выпуклое}, если:
	
	$\forall x_1, x_2 \in M$ и $\forall \lambda_1 , \lambda_2 \ge 0 : \lambda_1 + \lambda_2 = 1$ выполняется:
	
	$\lambda_1 x_1 + \lambda_2 x_2 \in M$
	
\subsection*{Определение 2.29}
	Выпуклая комбинация элементов $x_1, ..., x_n \in X$ называется $\lambda_1 x_1 + ... + \lambda_n x_n$, где 
	$\lambda_i \ge 0$ и $\lambda_1  + ...+  \lambda_n = 1$






\textbf{Упражнение:}
	Mножество M - выпуклоe $\Leftrightarrow$ любая его выпуклая комбинация принадлежит $M$.
	
\textbf{Доказательство:}
	
	($\Leftarrow$) Частный случай $n = 2$ дает нам требуемое.
	
	($\Rightarrow$)(feat. Bakel) Пусть $M$ выпуклое множество, тогда
	
	$\forall x_i, x_{i+1}\in M$ и $\forall \lambda_i , \lambda_{i+1} \ge 0 : \lambda_i + \lambda_{i+1} = 1$ выполняется:	
$$
\begin{cases}
  	\lambda_1 x_1 + \lambda_2 x_2 \in M \\
    \lambda_2 x_2 + \lambda_3 x_3 \in M \\ 
    ... \\
    \lambda_{n-1} x_{n-1} + \lambda_n x_n \in M \\ 
\end{cases}
$$	
Просуммируем и получим линейную комбинацию, принадлежащую $M$
	
	$\lambda_1 x_1 + 2\lambda_2 x_2 + 2\lambda_3 x_3 + ... + \lambda_n x_n$
	
	Рассмотрим коэффициенты:
	
	$\lambda_1  + 2\lambda_2 + 2\lambda_3 ...+  \lambda_n = n - 1$
	
	Тогда:
	
	$\frac{1}{n-1}(\lambda_1  + 2\lambda_2 + 2\lambda_3 ...+  \lambda_n) = 1 \in M$
	
	Пусть $\lambda_i' = \frac{1}{n-1}\lambda_i$, тогда
	
	$\lambda_1' x_1  + \lambda_2' x_2 + \lambda_3' x_3 + ...+  \lambda_n' x_n \in M$ 
	
	В силу произвольности выбора $\lambda_i$ мы рассмотрели все комбинации.
	
	
	
	
	
	
	
	
\section{Нормированные пространства}
		Линейное пространство $X$ - нормированное, если в нем определено отображение $\phi(x) = ||x||$ : $ x \to R_+$ со свойствами:
		
		1) $\forall x \in X: ||x|| \ge 0$, $||x|| = 0 \Leftrightarrow x = \Theta$
		
		2) $\forall \lambda \in R:  || \lambda x || = |\lambda|||x||$
		
		3) $\forall x,y \in X : ||x + y || \leq ||x|| + ||y||$
	
		\textbf{Каноническая метрика:} $\rho(x,y) = ||x-y||$
		
		Всякое нормированное пространство является метрическим(наооборот не верно). Значит и все аксиомы ЛП справедливы.
		
\subsection*{Определение 2.30}
	
	Последовательность ${x_n}$ из метрического пространства \textbf{сходится к $x_0$}, если:
	
	$$\lim_{n \to \infty} x_n = x_0 \Leftrightarrow \lim_{n \to \infty} ||x_n - x_0|| = 0$$
	
\subsection*{Свойство непрерывности нормы}\label{eq100}
	Пусть последовательность ${x_n}$ элементов нормированного пространства X сходится к элементу $x_0$ того же пространства. Тогда выполняется	
	$$\lim_{n \to \infty} ||x_n|| = ||x_0||$$
		
\textbf{Доказательство}		
	Оценочки:
	
	$||x_n|| = ||x_n + x_0 - x_0|| \leq ||x_n - x_0|| + ||x_0||$
	
	$||x_0|| = ||x_0 + x_n - x_n|| \leq ||x_0 - x_n|| + ||x_n||$
	
	Отсюда, в силу сходимости ${x_n}$к $x_0$:
	
	$||x_n|| - ||x_0|| \leq ||x_n - x_0|| \to 0 $
	
	$||x_n|| - ||x_0|| \ge -||x_n - x_0|| \to 0$
	
	По теореме о двух милиционерах получаем: 	$$\lim_{n \to \infty} ||x_n|| = ||x_0||$$
	
	Мы попали в определении предела функции $f(x) = ||x||$ по \hyperref[eq201]{Гейне}, значит \textbf{доказано}
	
\subsubsection*{Определение 2.31}
	Пусть $X$ - линейное пространство и есть две нормы $||x||_1, ||x||_2$
	
Тогда $||x||_1 \sim ||x||_2$ - эквивалентны, если существуют $c_1, c_2 > 0$ и $\forall x \in X$:

	$$c_1||x||_1 \leq ||x||_2 \leq c_2||x||_1$$
		
\textbf{Упражнение}		
	$\sim$ - отношение эквивалентности

\textbf{Доказательство:}

	\textbf{1) Рефлексивность:}
		
		$c_1||x|| \leq ||x|| \leq c_2||x||$ - верно, при $c_1 = c_2 = 1$
		
	\textbf{2)	Симметричность:}
	
		 $\exists c_1, c_2$:
		$c_1||x||_1 \leq ||x||_2 \leq c_2||x||_1$
	
$$
\begin{cases}
   c_1||x||_1 \leq ||x||_2 \\
   c_2||x||_1  \ge ||x||_2  
\end{cases}
\Rightarrow
\begin{cases}
   ||x||_1 \leq \frac{1}{c_1}||x||_2 \\
   ||x||_1  \ge \frac{1}{c_2}||x||_2  
   
\end{cases}
\Rightarrow
\frac{1}{c_2}||x||_2 \leq ||x||_1 \leq \frac{1}{c_1}||x||_2
$$

Тк $\frac{1}{c_1} , \frac{1}{c_2} > 0$, то доказано

			
	\textbf{3)	Транзитивность:}
	Пусть есть три нормы $||x||_1 , ||x||_2 , ||x||_3$:
	
	Если $||x||_2 = 0 \Rightarrow ||x||_3 = 0$, короче все норм.
	
	Если $||x||_2 \neq 0 $
		$$
\begin{cases}
	c_1||x||_1 \leq ||x||_2 \leq c_2||x||_1 \\
	d_1||x||_2 \leq ||x||_3 \leq d_2||x||_2
\end{cases}
\Rightarrow
	d_1 c_2 ||x||_1 \leq ||x||_3  \leq c_2 d_2||x||_1
$$
		Доказано
		
\subsection{Теорема об эквивалентности норм в КЛП}
	В конечномерном линейном пространстве $X$ все нормы эквивалентны
	
\textbf{Доказательство:}
	
	Тк ЛП $X$ - конечномерное, то в нем есть базис $e_1 , e_2, ..., e_n$
		
	\textbf{Евклидова норма:} $||x||_e = \sqrt{\alpha_1^2 + \alpha_2^2 + ... +\alpha_n^2  }$
	
	Докажем, что $\forall ||x|| : ||x|| \sim ||x||_e$ и в силу транзитивности будет доказано.
	
\textbf{	Верхняя оценка}
	Раскладывая по базису, получим:
	
	$||x|| = || \alpha_1 e_1 + ... + \alpha_n e_n|| \leq |\alpha_1|$ $||e_1|| + ... + |\alpha_n|$ $||e_n|| \leq $
	
	$\leq max(||e_k||) (|\alpha_1| + |\alpha_2| + ...+ |\alpha_n|)$
	
	Применяя неравенство Коши-Буняковского($a_i = |a_i|, b_i = 1$):
			
	$max(||e_k||) (|\alpha_1| + |\alpha_2| + ...+ |\alpha_n|) \leq  \sqrt{n}$ $max(||e_k||)\sqrt{\alpha_1^2 + \alpha_2^2 + ... +\alpha_n^2  } \leq$
	
	$\leq n $ $max(||e_k||)\sqrt{\alpha_1^2 + \alpha_2^2 + ... +\alpha_n^2  } = N_1\sqrt{\alpha_1^2 + \alpha_2^2 + ... +\alpha_n^2  } = N_1 ||x||_e$

\textbf{Нижняя оценка}

		Рассмотрим функцию $f(x): f(x) = ||x||$ - непрерывная, по \hyperref[eq100]{теореме о непрерывности нормы.}
		
		 Обозначим $S = \{ x \in X : ||x||_e = 1 \}$. Множество S можно эквивалентным образом рассматривать как сферу $S_1(0)$ единичного радиуса с центром в
точке $(0,..., 0)$ евклидова пространства $R^n$

	Тк $S_1(0)$ - ограниченное и замкнутое ,\hyperref[eq202]{по теореме Вейерштрасса} функция $f(x)$ достигает в нем
	своего минимального и максимального значения, а значит:
	$$\exists x_0 \in S: \forall x \in S: f(x) \ge f(x_0) = ||x_0|| > 0$$, тк все значения лежат на сфере ненулевого радиуса.
		
		Пусть $x \in X$, 
		
		$$f(x) = ||x|| = ||\frac{x}{||x||_e}|| \cdot ||x||_e $$
		
		$\frac{x}{||x||_e} \in S$, тк мы нормировали вектор $x$ и его модуль стал равен 1 (те Евклидова норма равна 1).
		
	Значит можно оценить:
		 
	$$||\frac{x}{||x||_e}| \ge ||x_0|| = N_0, x_0 \in S$$
	Отсюда:
		 
	$||x|| \ge N_0 ||x||_e$
		 
 	\textbf{Оценка снизу доказана}
		
	Ну и получили определение эквивалентности норм
	$N_0||x||_e \leq ||x|| \leq N_1||x||_e$
	, в силу транзитивности и произвольности $||x||$ все нормы будут эквивалентны.

\section{Фактор-пространство}

\subsection*{Определение 2.31}
	\textbf{Подпространство линейного пространства} - линейное пространство относительно тех же операций + замкнутое множество относительно операций изначального пространства.
	
\subsection*{Определение 2.32}	\label{eq102}
	Пусть $X$ есть линейное векторное пространство, $L$ — его подпространство, $x \in X$. \textbf{Классом смежности $\pi(x)$ называется множество} 
	$$\pi(x) = \{ y \in X : y = x + z, z \in L \}.$$
	
	\textbf{Фактор-пространство} $X\text{/}_L$ - совокупность всех $\pi(x), x \in X$
	
\subsection{Теорема о классах смежности}
		
	Любые два фактор-класса или совпадают, или не пересекаются.
	
\textbf{Доказательство:}

	Пусть $\pi(x_1) \cap \pi(x_2) \neq \varnothing$.
	Тогда $\exists y_0 \in \pi(x_1) \cap \pi(x_2)$ и выполняется:
 	
 	$y_0 = x_1 + z_1 = x_2 + z_2$, где $z_1, z_2 \in L$
 	
 	Отсюда:
 	
 	$x_1 - x_2 = z_2 - z_1 \in L	 \Rightarrow x_1 - x_2 \in L$
	
	 1) Пусть $y$ есть произвольный элемент из $\pi(x_1)$. Имеет место цепочка равенств
	  $$y = x_1 + z = x_2 + ((x_1 - x_2) + z) $$

Поскольку $z, x_1 - x_2 \in L$, то получаем, что $y$ есть элемент $\pi(x_2)$. 
Таким образом, всякий элемент y из класса $\pi(x_1)$ принадлежит классу $\pi(x_2)$.

	2) Можно доказать и обратно:

	$y_0 = x_1 + z_1 = x_2 + z_2 \Rightarrow x_2 - x_1 = z_1 - z_2 \in L	\Rightarrow x_2 - x_1 \in L$
	
		 Пусть $y$ есть произвольный элемент из $\pi(x_2)$. Имеет место цепочка равенств
	  $$y = x_2 + z = x_1 + ((x_2 - x_1) + z) $$
	
Поскольку $z, x_2 - x_1 \in L$, то получаем, что $y$ есть элемент $\pi(x_1)$. 
Таким образом, всякий элемент $y$ из класса $\pi(x_2)$ принадлежит классу $\pi(x_1)$.	
	\textbf{Доказано}
	
	
	
	
\section{Изометрия, изоморфизм пространств}

\subsection*{Определение 2.33}
	Метрические пространства $(X , \rho_1) , (Y , \rho_2)$ называются \textbf{изометричными}, если существует биекция  	
	$J(x) : X \to Y : \forall x_1,x_2 \in X$ выполнятся:
		
	$\rho_1(x_1,x_2) = \rho_2(J(x_1) , J(x_2))	$	



\subsection*{Определение 2.34}
Линейные пространства $X$ и $Y$ называются \textbf{изоморфными}, если
существует биекция $J(x) : X \to Y$, что $ \forall x_1, x_2 \in  X$ и $\forall \lambda , \mu \in R$ выполняется:
	$$J(\lambda x_1 + \mu x_2) = \lambda J(x_1) + \mu J(x_2)$$
	
	
\subsection{Определение 2.35}	
	Нормированное пространство $X$ называется \textbf{вложенным} в нормированное пространство $Y$ , если 			   	существует отображение $J(x)$,	
	определенное на всем $X$ и такое, что

	1) $J(\lambda x_1 + \mu x_2) = \lambda J(x_1) + \mu J(x_2), x_1, x_2 \in X, \lambda, \mu \in R$

	2) $\exists M \ge 0 : ||J(x)||_Y \ge M||x||_X$ $\forall x \in X$
	
\subsection{Определение 2.36}
	Последовательность элементов из метрического нормированного пространства $(X,\rho)$ фундаментальная, если
	
	$\forall \varepsilon > 0$ $\exists N = N(\varepsilon) : ||x_n - x_m|| < \varepsilon$ $\forall n, m > N; x_n, x_m \in X$  
	
\subsection{Определение 2.37}
	\textbf{Полное пространство} - пространство, в котором любая фундаментальная последовательности сходится
	к элементу этого же пространства.

\subsection{Определение 2.38}
	Нормированное пространство, полное по метрике $||x - y||$, называется \textbf{банаховым} пространством.	

\section{Нормируемость фактор-пространства}	
	
\subsection{Теорема о замкнутых классах смежности}

	Если $X$ есть нормированное пространство, то любой класс смежности есть замкнутое множество.
	
	\textbf{Доказательство:}
	Пусть $x \in X , \pi(x)$ - его класс смежности. 
	Пусть $\{ y_n \}_{n = 1}^{\infty}$ последовательность элементов из $\pi(x)$, сходящаяся к $y_0$, тогда
	
	$$y_n = x + z_n , z_n \in L $$
	
	Тк $y_0 \in X, x \in X$, то справедливо 
	$$y_0 = x + z_0$$
	
	Заметим, что если сходится  $\{ y_n \}_{n = 1}^{\infty}$ к $y_0$, то сходится $\{ z_n \}_{n = 1}^{\infty}$ к элементу $z_0$. 

 Поскольку подпространство $L$ есть замкнутое множество,
то $z_0$ есть элемент $L$, а значит по определению класса смежности $y_0 \in \pi(x)$. Получается любой класс смежности содержит все свои предельные точки, по определению \hyperref[eq101]{замкнутого множества} \textbf{Доказано}
	
\section{Теорема о нормируемости фактор-пространства}
	Если X есть нормированное пространство,
то и \hyperref[eq102]{фактор-пространство} $X\text{/}_L$ будет нормируемым.
		
\textbf{Доказательство:} 

	Путь $$||\pi(x)||_{X\text{/}_L} = inf_{y \in \pi(x)} ||y|| $$
	
Тупа докажем аксиомы нормированного пространства.

1) 	
	
	
	
	
\section{Теорема о подпространстве банахового пространства}

	Пусть $X$ есть банахово пространство, $L$ — его подпространство, 
	
	${X\text{/}_L}$  фактор-пространство 	с нормой:
	

		$$||\pi(x)||_{X\text{/}_L} = inf_{y \in \pi(x)} ||y|| $$
Тогда $X\text{/}_L$ есть банахово пространство.
	
		
	\textbf{Без доказательства}
	
	
\section{Гильбертовы пространства}

\subsection{Определение 2.39}	
	Линейное пространство называется \textbf{унитарным}, если в нем определено скалярное произведение со свойствами:
	
	$\forall x,y,z \in X$, $\forall \lambda \in Q$	
	
	1) $(x, x) \ge 0, (x, x) = 0 \Leftrightarrow x = \Theta$
	
	2) $(x, y) = \overline{(y, x)}$
	
	3) $(\lambda x, y) = \lambda (x, y)$
	
	4) $(x + y, z) = (x, z)+(y, z)$


	
	
	
\newpage
\part{Полезности}
\subsection{Связная область}\label{eq4}
Определение. Пусть задана область $E$, т.е. множество, состоящее из внутренних точек. Множество $E$ называется связным, если любые две точки этого
множества можно соединить ломаной, целиком лежащей в этой области.

\subsection{Формула конечных приращений Лагранжа}\label{eq5}	
Если функция $F$ непрерывна на отрезке $[a,b]$  и дифференцируема в интервале $(a,b)$, то найдётся такая точка $ c\in (a,b)$, что
 
	$F(a) - F(b) = F'(c)(b - a)$
	
	Можно записать так:
	
	$F(x + \Delta x) - F(x) = F'(x + \theta\Delta x)\Delta x , \theta \in (0,1) $

\subsection{Теорема о смешанных производных}\label{eq6}
\includegraphics[width = 20cm]{CP.jpg}	

\subsection{Вывод $cos\gamma$ в т. Гаусса-Остроградского}\label{eq10}
Учитывая, что 
$$
\begin{cases}
   f = u \\
   g = v \\
   h = \psi(u,v)
\end{cases}
\Rightarrow
\vec{n} = 
\begin{vmatrix}
\vec{i} & \vec{j} & \vec{k} \\
(f_u = 1) & (g_u = 0) & h_u \\
(f_v = 0) & (g_v = 1) & h_v 
\end{vmatrix}
=
\-h_u \vec{i} - h_v \vec{j} + \vec{k}
$$
Отсюда:
$$
|\vec{n}| = \sqrt{1+h_u^2 + h_v^2}
$$

А значит:
$$
\vec{V} = ( ..., ... , \frac{1}{\sqrt{1+h_u^2 + h_v^2}})
=
( ..., ... , \frac{1}{\sqrt{1+\psi_x^2 + \psi_y^2}})
$$
те $cos\gamma = \frac{1}{\sqrt{1+h_u^2 + h_v^2}} = \frac{1}{\sqrt{1+\psi_x^2 + \psi_y^2}}> 0$, $\gamma$ - острый, а значит $\vec{V}$ - внешняя нормаль к $S_2$

Аналогично для $\varphi(x,y)$:
$$cos\gamma = \frac{1}{\sqrt{1 + \varphi_x^2 + \varphi_y^2 }} $$

\subsection{Замечание в т. Гаусса-Остроградского:}\label{eq11}
	В оригинальной методичке написано вот так:
	
	\includegraphics[width = 8cm]{ориг.jpg}	
		\begin{wrapfigure}{r}{6cm}
	\includegraphics[width = 6cm]{контрпример.jpg}
	\end{wrapfigure}
	
	Те почему-то считается, что $cos\gamma$ угол между нормалью $S_2$ и осью $Oz$ - с точностью до знака совпадает с углом между нормалью $S_1$ и осью $Oz$. Но это же абсурд. Можно привести контрпример, когда это не выполняется:
	

	$\vec{n_1}(0 , 0 , 1)$
	
	$\vec{n_2}(a , b , c), $ где $a,b \neq 0$

	Понятно, что нормаль к $S_2$ никогда не будет нормалью к $S_1$, ну и углы не совпадут соответственно.
	
	Но если мы говорим, что $S_1$ и $S_2$ - части кусочно-гладкой поверхности, то в силу того, что мы нормаль ищем в конкретной точке нашей поверхности, мы можем писать общий угол $\gamma$.
\newpage

\subsection{Предел по Гейне} \label{eq201}
	\includegraphics[width = 18cm]{предел_по_Гейне.jpg}

\subsection{Теорема Вейерштрасса} \label{eq202}

	\includegraphics[width = 18cm]{тВейерштрасса.jpg}
	
	
\end{document}






















