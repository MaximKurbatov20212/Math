\documentclass{article}
\usepackage{scrextend}
\usepackage{graphicx}
\usepackage{amssymb}
\usepackage{xcolor}
\usepackage{hyperref}
\usepackage{amsmath}
\definecolor{linkcolor}{HTML}{002f55} % цвет ссылок
\definecolor{urlcolor}{HTML}{002f55} % цвет гиперссылок
\hypersetup{pdfstartview=FitH,  linkcolor=linkcolor,urlcolor=urlcolor, colorlinks=true}
\changefontsizes[20pt]{14pt}
\usepackage[utf8]{inputenc}
\usepackage[left=20mm, top=20mm, right=10mm, bottom=20mm,  nohead,nofoot]{geometry}
\usepackage[russian]{babel}


\begin{document}
\tableofcontents
\setcounter{tocdepth}{4}
\newpage

\part{Интегрирование на многообразиях}
\section{Кривые в $R^n$}
Мы будем рассматривать наши кривые в пространстве $R^n$. Иногда в формулировке теоремы или утверждения нет условия на непрерывность кривой. Это не означает, что его нет, возможно оно и так подразумевается и без него утверждение становится интуитивно некорректным.
\subsection*{Определение 1.1:}
\begin{flushleft}
\textbf{Непрерывная кривая} — множество точек
$\varphi_1(t), ... , \varphi_n(t), t \in [a,b]$

$A = \varphi_1(a), ... , \varphi_n(a)$

$B = \varphi_1(b), ... , \varphi_n(b)$

Если A = B, то кривая замкнута.
\end{flushleft}
\subsection*{Определение 1.2:}
\begin{flushleft}
$\Phi(t) = (\varphi_1(t), ... , \varphi_n(t))$ — параметризация кривой

\textbf{Важный факт:} существует бесконечное кол-во способов параметризовать кривую
\end{flushleft}
\subsection*{Определение 1.3:}
\begin{flushleft}
Если для кривой выполнятеся:
$\exists \varphi_1'(t), ... , \varphi_n'(t)$ такие, что

$\varphi_1'^2(t)+ ... +\varphi_n'^2(t) > 0, t \in [a,b]$, то такую кривую называем \textbf{гладкой}

Если $\varphi_1'^2(t)+ ... +\varphi_n'^2(t) = 0,$ при $t = m$, то такая точка \textbf{особенная}
\end{flushleft}
\subsection*{Определение 1.4:}
\begin{flushleft}
\textbf{Кусочно-гладкая кривая} — \textbf{непрервыная} гладкая 
кривая, состоящая из \textbf{конечного} числа гладких кривых.

\textbf{Важный факт:} не каждая кривая является спрямляемой
\end{flushleft}
\subsection*{Определение 1.5:}
\begin{flushleft}
\textbf{Спрямляемая кривая} — кривая, имеющая конечную длину.

\textbf{Важный факт:} гладкая кривая всегда спрямляема

\end{flushleft}
\subsection*{Определение 1.6:}
\begin{flushleft}
\textbf{Натуральная параметризация} — параметризация, параметром которой выступает длина \textbf{дуги} от начала до точки на кривой.

Обозначаем ее как $\Psi(s)$ , где $s$ — длина дуги
\end{flushleft}
\begin{figure}[h]
\centering
\includegraphics[width = 8cm]{кривая.jpg}
\end{figure}

\subsection*{Теорема 1.1:}
	Для любой гладкой кривой существует натуральная параметризация.
	
	\textit{Без доказательства.}
	
\subsection*{Любопытное утверждение:}
Если кривая гладкая и без особых точек с гладкой параметризацией 
$\Phi(t)$ и натуральной параметризацией $\Psi(s)$ справедливо:
\begin{center}
$\frac{ds}{dt} = |\Phi'(t)|$
\end{center}
\newpage
\subsection*{Некоторые факты:}
Задание параметризации $(\varphi_1(t), ... , \varphi_n(t))$ определяет движение на кривой $\Gamma $от ее начальной точки к конечной, или, другими словами, определяет ориентацию кривой, называемую положительной. Если при переходе от исходной параметризации начальная и конечная точки меняются местами (в случае замкнутой кривой — меняется направление движения), то происходит смена ориентации от положительной к отрицательной. Кривую $\Gamma$ с положительной по отношению к исходной параметризации ориентацией обозначают $\Gamma^+$, с отрицательной — $\Gamma^-$.


\section{Криволинейные интегралы I рода}
\subsection*{Определение 1.7:}
	Пусть задана гладкая, спрямляемая кривая с параметризацией 
	$\Phi(t)$
	
	$\Gamma: \Phi(t) , t \in [a,b]$
	
	Также есть натуральная параметризация:
	
	 $\Gamma: \Psi(s),$
	$s \in [0 , S_\Gamma]$, в силу спрямляемости
	
	И пусть задана функция $F(x) , x \in \Gamma$
	
	Тогда \textbf{криволинейным интегралом I рода от $F$ по $\Gamma$}
	назовем интеграл Римана:
	
	$$ I = \int\limits_{0}^{S_\Gamma} F(\Psi(s)) \,ds  = 
	 \int\limits_{0}^{S_\Gamma} 	F(s) \,ds $$
	И будем обозначать его, как 
 	$$I = \int\limits_{\Gamma}^{} F_0(x) \,ds$$
\newpage
\subsection*{Свойства криволинейных интегралов I рода}
\subparagraph{Свойство 1:} $F(s) = 1 \Rightarrow I = S_\Gamma$

\textit{Док-во:}
	$$F(s) = 1 \Rightarrow I = \int\limits_{0}^{S_\Gamma} 1 \,ds \Rightarrow
	I = S_\Gamma - 0 = S_\Gamma$$ читд

\subparagraph{Свойство 2:} Криволинейный интеграл I рода не зависит от ориентации кривой, те $$\int\limits_{\Gamma^+}^{} F_0(x) \,ds = \int\limits_{\Gamma^-}^{} F_0(x) \,ds$$

\textit{Док-во:}


	Пусть дана кривая с натуральной параметризацией $\Psi(s),s \in [0 , S_\Gamma]$:
	
	$\Gamma^+: A = \Psi(0) , B = \Psi(S_\Gamma)$
	
	Возьмем точку $M \in [A , B]$ на кривой, тогда $M = \Psi(s)$
	
	Определим параметр $\sigma = S_\Gamma - s$, те $\sigma$ — расстояние от $B$ до $M$.Тогда

	$$\int\limits_{\Gamma^+}^{} F_0(x) \,ds = 
	\int\limits_{0}^{S_\Gamma} F(\Psi(s)) \,ds 
	\stackrel{\sigma = S_\Gamma - s}{=} 
	- \int\limits_{S_\Gamma}^{0} F(\Psi(\sigma - S_\Gamma)) \,d\sigma =  $$		 	
	$$=\int\limits_{0}^{S_\Gamma}F(\Psi(\sigma - S_\Gamma)) \,d\sigma = 
	\int\limits_{\Gamma^-}^{} F_0(x) \,d\sigma$$
Тк криволинейный интеграл I рода не зависит от выбранной параметризации, то свойство 2 доказано.
читд

\subparagraph{Свойство 3:}
Пусть $\Gamma$ есть кривая в $R^n$ с непрерывно дифференцируемой
на отрезке $[a, b$] параметризацией $\Phi(t)$ без особых точек, тогда справедливо равенство
$$\int\limits_{\Gamma}^{} F_0(x) \,ds = \int\limits_{a}^{b} F(\Phi(t))[\varphi_1'^2(t)+ ... +\varphi_n'^2(t)] ^ \frac{1}{2} \,dt $$
\textit{Без доказательства}

\subparagraph{Свойство 4:}
Пусть $\tau$ = $\{s_i\}_{i = 0}^{m}$ есть разбиение отрезка $[0, S_\Gamma]$, $\xi_i$  есть
точки из отрезков $[s_{i-1}, s_i], i = 1,...,m, \Delta s_i = s_i - s_{i-1}$  длина дуги
кривой $\Gamma$ от точки $\Psi_0(s_{i-1})$ до точки $\Psi_0(s_i)$, $\sigma_\tau$ — интегральная сумма функции $F_0(s)$ по отрезку $[0, S_\Gamma]$

$$ \sigma_\tau = \sum_{i=1}^m F_0(\Psi_0(\xi_i))\Delta s_i$$

Тогда, если криволинейный интеграл I первого рода существует, то
	\[ \lim_{max(\Delta s_i)\to 0} \sigma_\tau  = I \]
\textit{Док-во:}

	Вспомним, как мы определяли интеграл Римана. Мы составляли интегральные суммы, потом устремляли разбиение к нулю и говорили, если вот существует такой предел, то назовем его интегралом Римана. Тут у нас условие, что криволинейный интеграл I первого рода существует, значит существует интеграл Римана, значит и предел сумм есть, который как раз и равен нашему интегралу Римана.

\subparagraph{Свойство 5:}
Если функция $F(x)$ представляет собой комбинацию
$\alpha F_1(x) + \beta F_2(x)$, $\alpha, \beta$ — фиксированные числа, криволинейные интегралы по кривой $\Gamma$ от функций $F_1(x)$ и $F_2(x$) существуют, то выполняется
равенство.

	$$\int\limits_{\Gamma}^{} F_0(x) \,ds 
	= 
	\alpha \int\limits_{\Gamma}^{} F_1(x) \,ds + 
	\beta \int\limits_{\Gamma}^{} F_2(x) \,ds $$

\textit{Док-во:}
$$
	\int\limits_{\Gamma}^{} F_0(x) \,ds
=
 	\int\limits_{0}^{S_\Gamma} F(\Psi(s)) \,ds 
=
	\int\limits_{0}^{S_\Gamma} \alpha F_1(\Psi(s))+ \beta F_2(\Psi(s))\,ds =
$$
$$
	\int\limits_{0}^{S_\Gamma} \alpha F_1(\Psi(s)) \,ds + 
	\int\limits_{0}^{S_\Gamma} \beta F_2(\Psi(s)) \,ds 
= 
	\alpha \int\limits_{0}^{S_\Gamma}  F_1(\Psi(s)) \,ds +  
	\beta \int\limits_{0}^{S_\Gamma}  F_2(\Psi(s)) \,ds 
=
$$
$$
	\alpha \int\limits_{\Gamma}^{}  F_1(x) \,ds +  
	\beta \int\limits_{\Gamma}^{}  F_2(x) \,ds 
$$
читд
	
	Вообщем сводим криволинейный интеграл к интегралу Римана, а там эти свойства уже доказаны в прошлом семестре.

	
\subsection*{Определение 1.8:}	
\textbf{Криволинейным интегралом по кусочно-гладкой кривой} $\Gamma$ называется число

\begin{equation}\label{eq1}
\int\limits_{\Gamma_1}^{} F_0(x)\,ds  + \int\limits_{\Gamma_2}^{} F_0(x)\,ds
\end{equation}

если каждый из криволинейных интегралов по $\Gamma_1$ и $\Gamma_2$ существуют.
	
	
\textbf{Замечание} Поскольку понятие определенного интеграла
 $$\int\limits_{a}^{b} F(x)\,dx $$
по отрезку можно расширить — например, до несобственного интеграла от
неограниченных функций или по неограниченному промежутку — то и понятие криволинейного интеграла первого рода можно расширить, определив несобственный криволинейный интеграл первого рода, или же перейти
к какой-либо иной конструкции, расширяющей понятие обычного определенного интеграла.	
	\newpage	
	
\section{Криволинейные интегралы II рода}	
Пусть $\Gamma$ есть кривая, параметризованная непрерывно-дифференцируемой
на отрезке $[a, b]$ вектор-функцией $\Phi(t)$, и пусть эта кривая не имеет особых
точек. Тогда, во-первых, в каждой точке $\Phi(t)$ определена касательная к  $\Gamma$,
и, во-вторых, от параметризации $\Phi(t)$ можно перейти к эквивалентной ей
натуральной параметризации $\Psi(s$). Обозначим через $cos \alpha_k, k = 1,...,n,$
направляющие косинусы единичного вектора $\vec{l} = \vec{l}(t)$ касательной к $\Gamma$ в
текущей точке (другими словами, искомый вектор $\vec{l}$ задается равенством 
$\vec{l} = (cos \alpha_1,..., cos \alpha_n) $и $\alpha_k, k = 1,...,n,$ есть углы между вектором $\vec{l}$
и положительным направлением соответствующей оси $Ox_k)$. 

\subsection*{Определение 1.9:}	
	Пусть задана функция $F_0(x)$, определенная при $x \in \Gamma$, 

и пусть $ F_0(s)$ = $F(\Psi(s))$.(Я тут переобозначил $F$ и $F_0$,так как это было в интегралах I рода, ибо можно запутаться, когда $F$ и $F_0$  меняются местами просто так)

Тогда \textbf{криволинейным интегралом второго рода по кривой $\Gamma$ от функции F(x)} по координате $x_k,$ $ k = 1,...,n,$ называется интеграл:

	$$I = \int\limits_{\Gamma}{} F_0 * cos \alpha_k d{s},$$
	если последний существует
	
	Обозначают как:
	
		$$I = \int\limits_{\Gamma}{} F_0 d{x_k},$$
	\newpage
\subsection*{Определение 1.10:}
	Область $G$ из пространства $R^2$ называется \textbf{элементарной относительно оси $Oy$}, если ее граница состоит из графиков двух непрерывных функций $\phi(x)$ и $\psi(x)$, определенных при
$x \in [a, b]$ и таких, что $\phi(x)	\leq \psi(x)$ для всех $x$, а также, быть
может, из некоторых отрезков прямых $x = a$ и $x = b$.


\includegraphics[width = 8cm]{oy.jpg}
\includegraphics[width = 8cm]{oy_1.jpg}

\subsection*{Определение 1.11:}
	Область $G$ из пространства $R^2$ называется \textbf{элементарной относительно оси $Ox$} областью, если ее граница состоит из графиков двух непрерывных функций $\alpha(y)$ и $\beta(y)$, определенных при
$y \in [c, d]$ и таких, что $\alpha(y)	\leq \beta(y)$ для всех $y$, а также, быть
может, из некоторых отрезков прямых $y = c$ и $y = d$.


\includegraphics[width = 8cm]{ox.jpg}
\includegraphics[width = 8cm]{ox_1.jpg}
\newpage
\subsection*{Замечание:}
	Группы Шваб идут по Крючковичу, у которого такие области называются \textbf{простыми} и, к тому же, оси меняются местами.
	
	
\includegraphics[width = 18cm]{крючкович_x.jpg}

\includegraphics[width = 18cm]{крючкович_y.jpg}
\newpage
\subsection*{Формула Грина:}
	Пусть $D$ есть ограниченная область из пространства $R^2$
с кусочно-гладкой границей $\Gamma$, ориентированной положительно, и пусть
эту область можно разбить на конечное число непересекающихся элементарных областей с кусочно-гладкими положительно-ориентированными границами. Далее, пусть $P(x, y)$ и $Q(x, y)$ есть заданные функции
такие, что 

1)  $P$ и $Q$ непрерывны в замкнутой области $D$


2) $P$ и $Q$ имеют непрерывные частные производные $\frac{\partial Q}{\partial x},\frac{\partial P}{\partial y} $ в замкнутой $D$

тогда верна формула Грина:

\begin{equation}\label{eq3}
	\int\limits_D \int (\frac{\partial Q}{\partial x} - \frac{\partial P}{\partial y}) dxdy =
	\oint\limits_{\Gamma} (Pdx + Qdy)
\end{equation}
\textit{Док-во для элементарной(и по $Ox$, и по $Oy$) $D$:} 

	Сведем двойной интеграл к повторному и применим формулу Ньютона-Лейбница:
$$
\int\limits_D \int \frac{\partial Q}{\partial x} dxdy = 
\int\limits_{c}^d dy \int\limits_{\alpha(y)}^{\beta(y)}\frac{\partial Q}{\partial x}dx = 
\int\limits_{c}^d  Q(x,\beta(y) - Q(x,\alpha(y))dy =
$$

$$
\int\limits_{c}^d  Q(x,\beta(y))dy -\int\limits_{c}^d Q(x,\alpha(y))dy
$$  
\includegraphics[width = 10cm]{грин.jpg}	

Мы можем параметризовать наши кривые 

	$
	\gamma_1: \alpha(t) , t \in [c , d].
	$
	
	$
	\gamma_2: \beta(t) , t \in [c , d].
	$
	
	$
	\gamma_3:  y = c, x = t , t \in [\alpha(c) , \beta(c)].
	$
	
	$
	\gamma_4:  y = d, x = t , t \in [\alpha(d) , \beta(d)].	
	$
	
	


Перепишем наши интегралы как криволинейные. Не забываем, что есть разница в направлении кривой!!!

$$
\int\limits_{\gamma_1}  Q(x,y)dy -\int\limits_{\gamma_2^-} Q(x,y)dy
= \int\limits_{\gamma_1}  Q(x,y)dy + \int\limits_{\gamma_2} Q(x,y)dy
$$  
Заметим, что
$$ 
\int\limits_{\gamma_3}  Q(x,y)dy = \int\limits_{\gamma_4} Q(x,y)dy = 0
$$
У нас получился интеграл по замкнутому контуру
$$ 
\int\limits_{\gamma_1}  Q(x,y)dy + \int\limits_{\gamma_2} Q(x,y)dy +
\int\limits_{\gamma_3}  Q(x,y)dy = \int\limits_{\gamma_4} Q(x,y)dy = 
\oint\limits_{\Gamma} Qdy
$$
$$
\int\limits_D \int\frac{\partial Q}{\partial x}dxdy = 
\oint\limits_{\Gamma} Qdy
$$

Аналогично:
$$
\int\limits_D \int\frac{\partial P}{\partial y}dxdy = 
-\oint\limits_{\Gamma} Pdx
$$
Складывая, получаем:
$$
\int\limits_D \int( \frac{\partial Q}{\partial x} - \frac{\partial P}{\partial y} )dxdy =
\oint\limits_{\Gamma} (Pdx + Qdy)
$$
читд


\textit{Док-во, если $D$ состоит из мн-ва непересекающихся, ненулеых элементарных областей:}

$
D = D_1 \cup D_2 \cup ... \cup D_m $ при $ D_i \cap D_j \neq  \varnothing, i \neq j
$
В силу свойства аддитивности двойного интеграла и факта, что граница области имеет нулевую меру:
$$
\int\limits_D \int (\frac{\partial Q}{\partial x} - \frac{\partial P}{\partial y}) dxdy
=
\sum_{i=1}^{m} \int\limits_{D_i} \int (\frac{\partial Q}{\partial x} - \frac{\partial P}{\partial y}) dxdy
$$
Применяя теперь для каждого слагаемого в правой части данного равенства доказанную выше формулу, получим
$$
\int\limits_D \int (\frac{\partial Q}{\partial x} - \frac{\partial P}{\partial y}) dxdy
= 
\sum_{i=1}^{m} \oint\limits_{D_i} (Pdx + Qdy) 
$$
В сумме, стоящей справа, содержатся интегралы по положительно ориентированным частям границ областей $D_i$, составляющим в целом границу $D$,
а также содержатся интегралы по тем частям границ областей  $D_i$, которые
лежат внутри  $D$, причем эти интегралы берутся дважды по одинаковым
кривым, но с противоположной ориентацией — в силу свойств криволинейных интегралов второго рода они взаимно уничтожаются. В результате
суммирования как раз и получится требуемое равенство.

читд
\subsection*{Замечание:}
Может возникнуть вопрос, что это за странная запись такая? 
$$
\oint\limits_{\Gamma}(Pdx + Qdy)
$$

Ведь у нас никогда не было, что разные функции интегрируются по разным переменным в одном интеграле. Можно это понимать так: Мы хотим вычислить силу, поэтому интегрируем работу по составляющим, где $P$ $x$-составляющая, $Q$ $y$-составляющая.
	
Или просто воспринимайте его как сумму:
$$
\oint\limits_{\Gamma} (Pdx + Qdy) =
\oint\limits_{\Gamma}Pdx + \oint\limits_{\Gamma} Qdy
$$

\subsection*{Замечание:}
	Не обязательно писать именно интеграл по замкнутой кривой, можно просто интеграл. Просто два нулевых интеграла нам дают такую возможность.
\newpage

\subsection*{Определение 1.12:}
	Зададим $(m+1)$ гладкие, замкнутые кривые $\Gamma_0 , \Gamma_1$ ...$\Gamma_m $
	
	Пусть $\Gamma_0$ - граница области $G$ 
	и $\Gamma_i \cap \Gamma_j = \varnothing$ при $ i \neq j$
	
	$\Gamma_i$ - граница области $G_i$, $\Gamma_i \in G$ ,  $i = 1$ ...$ m$
	
	Тогда $G \setminus (G_1 \cup G_2 \cup$...$\cup G_m)$ - $(m+1)$-связная область
	
\includegraphics[width = 10cm]{связная_область.jpg}

Заметим, что при таком задании ориентации границы $(m + 1)$-связной
области кривые $\Gamma_i$, $i = 1,...,m$, будут ориентированы отрицательно по
отношению к ограниченным областям $G_i$, кривые же $\Gamma_i^-$ , наоборот, будут
положительно ориентированы по отношению к $G_i$.

\subsection*{Формула Грина для многосвязных областей:}
	Пусть область $G$ $(m + 1)$-связна, ее внешний и внутренние контуры $\Gamma_0$, $\Gamma_1,$...$, \Gamma_m$ являются замкнутыми кусочно-гладкими
кривыми без самопересечений, и пусть граница области $G$ положительно ориентирована. Далее, пусть $P(x, y)$ и $Q(x, y)$ есть заданные функции
такие, что

1)  $P$ и $Q$ непрерывны в замкнутой области $G$


2) $P$ и $Q$ имеют непрерывные частные производные $\frac{\partial Q}{\partial x},\frac{\partial P}{\partial y} $ в замкнутой $G$

Тогда имеет место равенство
\begin{equation}\label{eq3}
\int\limits_G \int (\frac{\partial Q}{\partial x} 
-
\frac{\partial P}{\partial y}) dxdy 
=
\int\limits_{\Gamma_0} (Pdx + Qdy)
-
\sum_{i = 1}^m \int\limits_{\Gamma_i^-} (Pdx + Qdy)
\end{equation}

\textit{Док-во для двусвязной области $G$:}


Соединим область $G_1$ с кривой $\Gamma_0$ разрезом, который представляет собой кусочно-гладкую кривую без самопересечений. Обозначим разрез как $L$

\includegraphics[width = 8cm]{многосвязная.jpg}

Обозначим через $G^*$ область, полученную из $G$ удалением данного разреза, предполагая, что граница области $G^*$ состоит из границы $G$ (с сохранением ориентации) и разреза, проходимого дважды.Граница $G^*$ представляет собой кусочно-гладкую кривую, а значит по Формуле Грина для односвязной области имеем:

$$
\int\limits_{G^*} \int (\frac{\partial Q}{\partial x} 
-
\frac{\partial P}{\partial y}) dxdy 
=
\int\limits_{\delta G^*} (Pdx + Qdy)
$$	
Далее, поскольку двойной интеграл не меняется при присоединении к множеству интегрирования множества нулевой двумерной меры, то имеет место равенство
$$
\int\limits_{G^*} \int (\frac{\partial Q}{\partial x} 
-
\frac{\partial P}{\partial y}) dxdy 
=
\int\limits_{\delta G} (Pdx + Qdy)
= 
\int\limits_{G} \int (\frac{\partial Q}{\partial x} 
-
\frac{\partial P}{\partial y}) dxdy 
$$

Заметим, что $\delta G = \Gamma_0 \cup L \cup \Gamma_1$, вспомниая определение (\ref{eq1}) интеграла по кусочно-гладкой кривой:
$$
\int\limits_{\delta G} (Pdx + Qdy)
=
\int\limits_{\Gamma_0} (Pdx + Qdy)
+
\int\limits_{L^+} (Pdx + Qdy)
+
\int\limits_{L^-} (Pdx + Qdy)
+
\int\limits_{\Gamma_1} (Pdx + Qdy)
$$
Учитывая, что направление движения по кривой имеет значение:
$$
\int\limits_{G} \int (\frac{\partial Q}{\partial x} 
-
\frac{\partial P}{\partial y}) dxdy 
=
\int\limits_{\delta G} (Pdx + Qdy)
=
\int\limits_{\Gamma_0} (Pdx + Qdy)
-
\int\limits_{\Gamma_1^-} (Pdx + Qdy)
$$
Таким образом, мы получили формулу (\ref{eq3}) для случая двусвязной области $G$

Что будет в случае, если $G$ - $(m+1)$-связная область?
Да то же самое, только 
$\delta G = \Gamma_0 \cup (L_1 \cup$ ... $ \cup L_m)\cup (\Gamma_1 \cup$ ... $\cup\Gamma_m)$
$$
\int\limits_{\delta G} (Pdx + Qdy)
=
\int\limits_{\Gamma_0} (Pdx + Qdy)
+
\sum_{i = 1}^m (\int\limits_{L_i^+}(Pdx + Qdy) - \int\limits_{L_i^-} (Pdx + Qdy))
-
\sum_{i = 1}^m \int\limits_{\Gamma_i^-} (Pdx + Qdy)
$$
Отсюда немедленно получаем
$$
\int\limits_{\delta G} (Pdx + Qdy)
=
\int\limits_{\Gamma_0} (Pdx + Qdy)
-
\sum_{i = 1}^m \int\limits_{\Gamma_i^-} (Pdx + Qdy)
$$
читд

\subsection*{Теорема 1.2}

Пусть

1)  $P$ и $Q$ непрерывны в замкнутой, связной области $G$


2) $P$ и $Q$ имеют непрерывные частные производные $\frac{\partial Q}{\partial x},\frac{\partial P}{\partial y} $ в замкнутой $G$

тогда 4 свойства эквивалентны:


1) Независимость $P(x,y)$, $Q(x,y)$ от пути интегрирования в $G$

2) Для любой замкнутой кусочно-гладкой кривой $\Gamma$, целиком лежащей в $G$, выполняется
$$
\int\limits_{\Gamma} (Pdx + Qdy) = 0
$$

3) Существует функция $u(x, y$) такая, что для любых 
точек $(x, y)$ из $G$ выполняется

$$
du(x, y) = P(x, y) dx + Q(x, y) dy;
$$

4) Для любых точек $(x, y)$ из $G$ выполняется
$$
\frac{\partial Q(x,y)}{\partial x} = \frac{\partial P(x,y)}{\partial y} 
$$

\newpage
\textit{Док-во:} 
$(1 \Rightarrow 2)$
$$
\int\limits_{\Gamma_1} (Pdx + Qdy) 
=
\int\limits_{\Gamma_2} (Pdx + Qdy) 
$$
$$
\int\limits_{\Gamma_1} (Pdx + Qdy) 
-
\int\limits_{\Gamma_2} (Pdx + Qdy) 
=
0
$$
$$
\int\limits_{\Gamma_1} (Pdx + Qdy) 
+
\int\limits_{\Gamma_2^-} (Pdx + Qdy) 
=
0
$$
\includegraphics[width = 8cm]{4экв.jpg}	

В силу того, что $\Gamma = \Gamma_1 \cup \Gamma_2^-$(с учетом направления):
$$
\int\limits_{\Gamma} (Pdx + Qdy) 
=
\int\limits_{\Gamma_1} (Pdx + Qdy) 
+
\int\limits_{\Gamma_2^-} (Pdx + Qdy) 
=
0
$$

В силу произвольности выбора $\Gamma_1$ и $\Gamma_2$ получаем, что $\Gamma$ - тоже произвольная кривая.


$(2 \Rightarrow 1)$

	Пусть $\Gamma$ - замкнутая кусочно-гладкая кривая
	и выполняется $\int\limits_{\Gamma} (Pdx + Qdy) = 0$
	
Разобьем(с учетом направления) $\Gamma = \Gamma_1 \cup \Gamma_2$, где $\Gamma_1$ и $\Gamma_2$ - кусочно-гладкие или просто гладкие кривые.
Тогда:
$$
\int\limits_{\Gamma} (Pdx + Qdy) 
=
\int\limits_{\Gamma_1} (Pdx + Qdy) 
+
\int\limits_{\Gamma_2} (Pdx + Qdy) 
=
0
$$
Отсюда:
$$
\int\limits_{\Gamma_1} (Pdx + Qdy) 
=
\int\limits_{\Gamma_2^-} (Pdx + Qdy) 
$$
\newpage
$(1 \Rightarrow 3)$\textbf{Надо еще исправлять}
Пусть $M_0 = (x_0, y_0)$ есть фиксированная точка $G$, $M = (x^*, y^*)$ есть текущая точка $G$, $\Gamma : M_0M$ есть кусочно-гладкая кривая без самопересечений,
целиком лежащая в $G$ и соединяющая точки $M_0$ и $M$.
Пусть 
$$
u(x,y) = \int\limits_{M_0M} (Pdx + Qdy) 
=
\int\limits_{\Gamma} (Pdx + Qdy) 
$$
В силу условия \hyperref[eq4]{связности} $G$:
$ \exists h : (x^* + h , y^*) \in G$

Пусть прямая L, прямая соединяющая $M(x^* , y^*)$ и $M^*(x^* + h , y^*)$

Покажем, что 

$$u_x = \lim_{h \to 0}\frac{u(x^* + h , y^*) - u(x^* , y^*)}{h} = P(x^* , y^*) $$
\includegraphics[width = 13cm]{Т1.2.jpg}	

Имеем
$$
\frac{u(x^* + h , y^*) - u(x^* , y^*)}{h} 
= 
\frac{1}{h}(u(x^* + h , y^*) - u(x^* , y^*))
$$
$$
=
\frac{1}{h} \int\limits_{M}^{M^*} (Pdx + Qdy)
=
\frac{1}{h} \int\limits_{L} (Pdx + Qdy)
$$
Параметризуем отрезок L:

	$x = x^* + th , t \in [0,1] \Rightarrow dt = dx$
	
	$y = y^* \Rightarrow dy = 0$
$$
\int\limits_{L} (P(x,y)dx + Q(x,y)dy) = \int\limits_{0}^{1}P(x^* + th,y^*)dt
=
P(x^* + h,y^*) - P(x^*,y^*)
$$	
Применим формулу \hyperref[eq5]{конечных приращений Лагранжа}
$$
P(x^* + h,y^*) - P(x^*,y^*) = P(x^* + \theta h,y^*)h  ,\theta \in (0,1)
$$
Отсюда:
	
$$
P(x^* + \theta h,y^*) 
=
\frac{u(x^* + h , y^*) - u(x^* , y^*)}{h}
$$
Теперь при $h \to 0$	 получаем:

$$
P(x^* , y^*) 
=
u_x
$$
Аналогично доказываем, что $Q(x^* , y^*) = u_y$

$$
du(x^* , y^*) = u_x dx + u_y dy = P(x^* , y^*) dx + Q(x^* , y^*) dy
$$
Но нам нужно еще доказать дифференцируемость $u(x,y)$ в $G$

$u_x = P(x , y)$
$u_y = Q(x , y)$

Тк по условию у нас $P$ и $Q$ имеют непрерывные частные производные $\frac{\partial Q}{\partial x},\frac{\partial P}{\partial y} $ в замкнутой $G$, то существую вторые производные для $u(x,y)$, отсюда немедленно следует дифференцируемость$u(x,y)$
читд	
	
$(3 \Rightarrow 1)$	

\textbf{Одно звено} : $M_0(x_0 , y_0)$ и $M(x^* , y^*)$ 

\includegraphics[width = 13cm]{1.jpg}
$$
\int\limits_{\Gamma} (Pdx + Qdy) 
=
\int\limits_{\Gamma} (u_xdx + u_ydy) 
$$

Параметризуем $\Gamma$:

 $ x = \varphi(t)$
 
 $ y = \psi(t)$ , где $t \in [\alpha , \beta]$
$$
\int\limits_{\Gamma} (Pdx + Qdy)
=шо
\int\limits_{\alpha}^{\beta} P(\varphi'(t) , \psi'(t))*\varphi'(t)dt 
+ Q(\varphi'(t) , \psi'(t))*\psi'(t)dt 
=
$$
$$
\int\limits_{\alpha}^{\beta} \frac{d}{dt}(u(\varphi(t) ,\psi(t)))dt 
= 
u(\varphi(\beta) ,\psi(\beta)) -  u(\varphi(\alpha) ,\psi(\alpha))
$$
Получается, что интеграл зависит лишь от начальных точек, а значит не зависит от пути интегрирования

Если \textbf{n звеньев:}

\includegraphics[width = 13cm]{n.jpg}	

$$
u(x_1 , y_1) - u(x_0 , y_0) 
+ 
u(x_2 , y_2) - u(x_1 , y_1)
+ 
u(x_3 , y_3) - u(x_2 , y_2)
+ ... +
u(x^* , y^*) - u(x_{n-1} , y_{n-1})
=
$$
$$
=
u(x^* , y^*) - u(x_0 , y_0)
$$
Получается, что и от количества звеньев не завсисит

читд

$(3 \Rightarrow 4)$
$$
u_{xy}(x,y) = \frac{\partial P}{\partial y} = P_y
$$
$$
u_{yx}(x,y) = \frac{\partial Q}{\partial x} = Q_x
$$
В силу непрерывности $P_y , Q_y$, получается, что и
$u_{xy}(x,y), u_{yx}(x,y)$ непрерывны в $G$. А если если существуют смешанные непрерывные производные, то они равны.

\hyperref[eq6]{Теорема о смешанных производных}
$$
u_{xy} = u_{yx} \Rightarrow Q_x = P_y
$$

читд

$(4 \Rightarrow 2  \Rightarrow 3)$
	Пусть $P_y = P_y$

Рассмотрим кусочно-гладкую замкнутую кривую в замкнутой $G$

Тогда справедлива формула Грина:
$$
\int\limits_{\Gamma} (Pdx + Qdy)
=
\int\limits_{G} \int (\frac{\partial Q}{\partial x} 
- 
\frac{\partial P}{\partial y}) dxdy
=
0
$$
Отсюда получаем свойство 2:
$$
\int\limits_{\Gamma} (Pdx + Qdy)
=
0
$$
Ранее уже доказали, что $(2 \Rightarrow 3)$

читд

\section{Поверхности в $R^n$}
\subsection*{Опрделение 1.13 }

Пусть $G$ есть ограниченная область из пространства $R^2$,
$f(u, v),g(u, v), h(u, v)$ — определенные при $(u, v) \in G$ и непрерывные
на $G$ функции. \textbf{Непрерывной поверхностью $S$} называется множество:

$
S = \{(x, y, z) : x = f(u, v), y = g(u, v), z = h(u, v), (u, v) \in G \}
$

Вектор-функция $\Phi(u, v)= (f(u, v), g(u, v), h(u, v))$ называется
представлением, или \textbf{параметризацией} поверхности.	
	
\subsection*{Определение 1.14 }
	
Рассмотрим точку $(u_0 , v_0) \in S$
$$
(u_0 , v_0) = \begin{cases}
	\textbf{ не особая}, \text{если }  \Phi_u(u_0 , v_0), \Phi_v(u_0 , v_0) - \text{ЛН};\\
	\textbf{ особая}, \text{если } \Phi_u(u_0 , v_0) , \Phi_v(u_0 , v_0) - \text{ЛЗ};
\end{cases}
$$

\subsection*{Определение 1.15 }

Поверхность назывется \textbf{гладкой}, если все ее точки не особые.



\subsection*{Определение 1.16 }
Совокупность каcательных прямых к поверхности в точке -\textbf{касательная плоскость} к поверхности в этой точке.

\subsection*{Определение 1.17}
	
	Пусть задана поверхность $S$ и $M_0(x_0 , y_0 , x_0) \in S,  
$ где

$
x_0 = f(u_0 , v_0) , y_0 = g(u_0 , v_0) , 
z_0 = h(u_0 , v_0)$

Тогда \textbf{касательная к плоскости $S$ в $(u_0 , v_0)$} определяется через определитель данной матрицы:
\begin{center}
\begin{equation}
\includegraphics[width = 13cm]{kasatelnya.jpg}\label{eq7}
\end{equation}
\end{center}
\subsection*{Определение 1.18}
	Прямая, проходящая через точку касания поверхности с касательной плоскостью и перпендикулярная этой плоскости, называется \textbf{нормальной прямой к поверхности в указанной точке.}
	Нормаль определяется матрицей:
\begin{equation}
	\includegraphics[width = 13cm]{normal.jpg}\label{eq8}
\end{equation}
\subsection*{Замечание}
	У плоскости в точке есть две нормали, верная из них та, которая задается матрицей из определениия 1.17(\ref{eq8}).	
	
	
\textbf{Важный факт}
	В каждой точке гладкой поверхности S однозначно определена нормаль,вычисляемая по формуле (\ref{eq8}).
	
\subsection*{Определение 1.19}
Если на поверхности $S$ эта нормаль
меняется непрерывно, то поверхность $S$ называется \textbf{ориентированной}. При задании ориентации поверхности считается, что поверхность S является \textbf{ двусторонней }, и та сторона поверхности,
которая прилегает к нормали (\ref{eq8}), называется 
\textbf{ положительной стороной и обозначается $S^+$ }, противоположная же сторона называется 
\textbf{ отрицательной и обозначается $S^-$ }	.

Пример неориентированной поверхности: Лента Мебиуса
	

\subsection*{Определение 1.20}

Две поверхности $S_i$ и $S_j$ называются \textbf{соседними}, если кривые
$\Gamma_i$ и $\Gamma_j$ имеют одну или несколько общих дуг (общих участков,не вырождающихся в точку).

\subsection*{Определение 1.21 }
Поверхность называется \textbf{кусочно-гладкой}, если она состоит конечного числа гладких поверхностей, которые могут пересекаться лишь по своим граничным точкам.

Если $S$ - кусочно-гладкая поверхность, то ее можно представить ввиде:

	$S = S_1 \cup S_2 \cup ... \cup S_p$, где $S_i$ и $S_j$ или соседние или могут быть соеденены некоторой последовательностью поверхностей.
	
	Пример: Кубик
	
\includegraphics[width = 8cm]{cube.jpg}

\subsection*{Определение 1.22 }
Кусочно-гладкая поверхность S, состоящая из $m$ частей
$S_1$,...,$S_m$, называется \textbf{ориентируемой}, если существует такая ориентация кривых $\Gamma_1$, ..., $\Gamma_m$ (границ поверхностей $S_1,...,S_m$), что
части (дуги) этих кривых, принадлежащие двум различным кривым
$\Gamma_i$ и $\Gamma_j$, получают от них противоположную ориентацию.

\subsection*{I квадратичная форма поверхности}
	
	Введем обозначения:
		
	$E(u , v) = (\vec{\Phi}_u(u , v) , \vec{\Phi}_u( u, v)) = |\vec{\Phi}_u^2( u, v)|$ - квадрат скалярного произведения.
	
	$G(u , v) = \vec{(\Phi}_v(u , v) , \vec{\Phi}_v( u, v)) = |\vec{\Phi}_v^2( u, v)|$ - квадрат скалярного произведения.
	
	$F(u , v) = (\vec{\Phi}_u(u , v) , \vec{\Phi}_v( u, v))$
	
	Тогда:
	
	$$(d\vec{\Phi})^2 = (\vec{\Phi}_u du + \vec{\Phi}_v dv)^2 = Edu^2 + 2Fdudv + Gdv^2$$
	
	I квадратичная форма имеет вид:
	$$Edu^2 + 2Fdudv + Gdv^2$$			
	
\textbf{Замечание:}
	I квадратичная форма не отрицательна
	
\textit{Док-во:}
Как известно из курса Линала, квадрат скалаярного произведения неотрицателен.
	$$Edu^2 + 2Fdudv + Gdv^2 = (d\vec{\Phi})^2 \geqslant 0$$
	
\subsection*{Поверхностный интеграл I рода}
	Пусть $S : \Phi(u , v)$ - Гладкая поверхность и задана функция $\psi(u , v)$, тогда \textbf{поверхностным интегралом I рода от $\psi(u,v)$} назовем:
	
$$
\int\limits_{S} \psi(u , v) ds 
=
\int\limits_{\Omega}\int \psi(u,v)\sqrt{EG - F^2}dudv	
$$		
	
	Также введем меру:
$$
	\int\limits_{S}ds = mesS;
$$	
	
\subsection*{Свойства поверхностных интегралов I рода:}
	1) Линейность
$$
\int\limits_{S} (\alpha F_1 + \beta F_2)dS
=
\alpha\int\limits_{S}  F_1ds
+
\beta\int\limits_{S}  F_2ds
$$
	2)Аддитивность	
$$
\int\limits_{S}  F_2ds
+
\int\limits_{S}  F_1ds
	

\subsection*{Формула Гаусса-Остроградского:}
\textbf{Обозначаю: $R = R(x, y ,z) , Q = Q(x,y,z) , P = P(x,y,z)$;}

		Пусть $V$ есть ограниченная область из пространства $R^3$, и $V$ можно разбить кусочно-гладкими поверхностями на конечное число элементарных областей. Далее, пусть $P(x, y ,z)$ и $Q(x, y ,z)$ есть заданные функции
такие, что 

1)  $P$, $Q$ и $R$ непрерывны в замкнутой области $V$


2) $P$,$Q$ и $R$ имеют непрерывные частные производные $\frac{\partial Q}{\partial x},\frac{\partial P}{\partial y},
\frac{\partial R}{\partial z} $ в замкнутой $V$

Тогда верная формула Гаусса-Остроградского:
$$
\int\int\limits_{V}\int (\frac{\partial P}{\partial x} + \frac{\partial Q}{\partial y} + \frac{\partial R}{\partial z})dxdydz
=
\int\limits_{\delta V}\int (Pcos\alpha + Qcos\beta + Rcos\gamma)ds
$$
в которой $(cos\alpha, cos\beta, cos\gamma)$ есть направляющие косинусы вектора внешней нормали к границе $\delta V$ области $V$





























\newpage
\section{Полезности}
\subsection*{Связная область}\label{eq4}
Определение. Пусть задана область $E$, т.е. множество, состоящее из внутренних точек. Множество $E$ называется связным, если любые две точки этого
множества можно соединить ломаной, целиком лежащей в этой области.

\subsection*{Формула конечных приращений Лагранжа}\label{eq5}	
Если функция $F$ непрерывна на отрезке $[a,b]$  и дифференцируема в интервале $(a,b)$, то найдётся такая точка $ c\in (a,b)$, что
 
	$F(a) - F(b) = F'(c)(b - a)$
	
	Можно записать так:
	
	$F(x + \Delta x) - F(x) = F'(x + \theta\Delta x)\Delta x , \theta \in (0,1) $

\subsection*{Теорема о смешанных производных}\label{eq6}
\includegraphics[width = 20cm]{CP.jpg}	
\end{document}






















